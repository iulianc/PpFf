
\chapter{Les m\'ethodes publiques de l'API de \ppff}

\label{methodes-api.ann}


\KOMAoptions{paper=landscape,pagesize}
\recalctypearea


\begin{center}
\footnotesize
\begin{longtable}{|l|l|p{5cm}|}
%\caption{Les m\'ethodes publiques de l'API de~\ppff.\label{methodes_api.tab}}\\
\hline
\textbf{M\'ethode} & \textbf{Type du r\'esultat} & \textbf{Description du r\'esultat}\\
\hline
\endfirsthead
\multicolumn{3}{c}%
{\tablename\ \thetable\ Les méthodes publiques de l'API (\textit{suite})} \\
\hline
\textbf{M\'ethode} & \textbf{Type du r\'esultat} & \textbf{Description du r\'esultat}\\
\hline
\endhead
\hline \multicolumn{3}{r}{\textit{Suite page suivante}} \\
\endfoot
\hline
\endlastfoot
\hline
	\multicolumn{3}{|c|}{\textbf{Source} --- M\'ethodes {\bf statiques} pour cr\'eer un flux \`a partir d'une source\label{source.page}}
    \\
\hline
	\begin{tabular}{@{}l@{}}
	\tt Flow:$\!$:source(string\& path)
	\end{tabular} &
	\TT{Flow\&} & 
    Retourne un flux avec les lignes
    contenues dans le fichier indiqu\'e par \TT{path}.
    \\
\hline
	\begin{tabular}{@{}l@{}}
	\tt template<T, Iterator> \\
	\tt Flow:$\!$:source(Iterator  begin, Iterator end)
	\end{tabular} &
	\TT{Flow\&} &
	Convertit un conteneur de type {STL} en un flux.
    \\
\hline
	\multicolumn{3}{|c|}{\textbf{Transformation} --- M\'ethodes pour produire un flux \`a partir d'un flux existant\label{transformation.page}}
    \\    
\hline
	\begin{tabular}{@{}l@{}}
	\tt template<In> \\
	\tt find(Func<bool(In*)> const\& predicate)
	\end{tabular} &
  	\TT{Flow\&} &
    Retourne les
    \'el\'ements du flux qui satisfont \TT{predicate}.
    \\
\hline
	\begin{tabular}{@{}l@{}}
	\tt template<In, Container, Out> \\
	\tt flatMap(Func<Container*(In*)> const\& taskFunc)
	\end{tabular} &
  	\TT{Flow\&} & 
    Applique la fonction fournie en argument
    \`a chaque \'el\'ement du flux et concat\`ene ces \'el\'ements lorsque plusieurs sont produits par la fonction.
    \\
\hline
	\begin{tabular}{@{}l@{}}
	\tt template<In, Out, Container=In> \\
	\tt flatten()
	\end{tabular} &
  	\TT{Flow\&} &
    Aplanit un flux multi-niveaux en cr\'eant un flux unique \`a partir du contenu des divers flux/conteneurs.
    \\
\hline
	\begin{tabular}{@{}l@{}}
	\tt template<T> \\
	\tt limit(int n)
	\end{tabular} &
	\TT{Flow\&} & 
    Retourne un flux compos\'e des~\TT{n}~premiers \'el\'ements du flux d'entr\'ee.
    \\
\hline
	\begin{tabular}{@{}l@{}}
	\tt template<In, Out> \\
	\tt map(Func<Out*(In*)> const\& taskFunc)
	\end{tabular} &
	\TT{Flow\&} & 
    Retourne un flux compos\'e de
    l'application de \TT{taskFunc}
    \`a chacun des
    \'el\'ements du flux.
    \\
\hline
	\begin{tabular}{@{}l@{}}
	\tt template<In> \\
	\tt peek(Func<void(In*)> const\& taskFunc)
	\end{tabular} &
	\TT{Flow\&} &
	Applique la fonction \TT{taskFunc} \`a chaque \'el\'ement du flux et r\'e\'emet l'\'el\'ement (sans le modifier) sur le flux de sortie. Note: Utile pour le d\'ebogage.
    \\
\hline
	\begin{tabular}{@{}l@{}}
	\tt template<T> \\
	\tt skip(int n)
	\end{tabular} &
	\TT{Flow\&} &
    Retourne un flux compos\'e des \'el\'ements du flux d'entr\'ee, mais en omettant les \TT{n} premiers \'el\'ements.
    \\
\hline
	\multicolumn{3}{|c|}{\textbf{Aggregation} --- M\'ethodes qui produisent une valeur, typiquement scalaire, \`a partir des \'el\'ements d'un flux\label{collector.page}}
    \\     
\hline
	\begin{tabular}{@{}l@{}}
	\tt template<T> \\
	\tt allMatch(Func<bool(T*)> predicate)
	\end{tabular} &
  	\TT{bool} &
    Retourne \TT{true} si tous les \'el\'ements
    du flux satisfont \TT{predicate}, sinon \TT{false}.
    \\
\hline
	\begin{tabular}{@{}l@{}}
	\tt template<T> \\
	\tt anyMatch(Func<bool(T*)> predicate)
	\end{tabular} &
  	\TT{bool} & 
    Retourne \TT{true} si au moins un  
    \'el\'ement du flux satisfait \TT{predicate}, sinon \TT{false}.
\\          
\hline
	\begin{tabular}{@{}l@{}}
	\tt count()\\
	\end{tabular} &
  	\TT{unsigned int} & 
    Retourne le nombre d'\'el\'ements
    du flux.
    \\ 
\hline
	\begin{tabular}{@{}l@{}}
	\tt template<T> \\
	\tt max(Func<void(T*, T*)> compare)
	\end{tabular} &
	\TT{T} &
	Retourne l'\'el\'ement maximum du flux en fonction du comparateur.
    \\
\hline
	\begin{tabular}{@{}l@{}}
	\tt template<T> \\
	\tt min(Func<void(T*, T*)> compare)
	\end{tabular} &
	\TT{T} &
	Retourne l'\'el\'ement minimum du flux en fonction du comparateur.
    \\
\hline
	\begin{tabular}{@{}l@{}}
	\tt template<T> \\
	\tt noneMatch(Func<bool(T*)> predicate)
	\end{tabular} &
	\TT{bool} &
    Retourne \TT{true} si aucun des \'el\'ements
    du flux ne satisfait \TT{predicate},
    sinon \TT{false}.
    \\    
\hline
	\begin{tabular}{@{}l@{}}
	\tt template<In, Out=In> \\
	\tt reduce(Reducer<In, Out> const\& reducer)
	\end{tabular} &
	\TT{Out} &
	Effectue une r\'eduction sur les \'el\'ements du flux. Voir la notion de \TT{Reducer}, Section~\ref{reducer.sect}.
    \\
\hline
	\begin{tabular}{@{}l@{}}
	\tt template<In, Out=In> \\
	\tt reduce(Out init, Func<Out(In, Out)> acc)
	\end{tabular} &
	\TT{Out} &
	Effectue une r\'eduction des \'el\'ements du flux, en utilisant \TT{init} comme valeur initiale et \TT{acc} comme fonction d'accumulation.
    \\    
\hline
	\begin{tabular}{@{}l@{}}
	\tt template<T> \\
	\tt sum()
	\end{tabular} &
	\TT{T} &
	Retourne la somme des \'el\'ements du flux.
    \\
\hline
	\multicolumn{3}{|c|}{\textbf{Aggregation} --- M\'ethodes qui produisent une collection \`a partir d'un flux}
    \\     
\hline
	\begin{tabular}{@{}l@{}}
	\tt template<T, Container<T>{>}\\
	\tt collect()
	\end{tabular} &
  	\TT{Container<T>} &
    Retourne un conteneur
    STL avec tous les \'el\'ements du flux.
    \\
\hline
	\begin{tabular}{@{}l@{}}
	\tt template<In, K=In, V=In, MapType> \\
	\tt groupByKey(Func<K*(In*)> fk, Func<V*(In*)> fv)
	\end{tabular} &
  	\TT{MapType} &
    Retourne un dictionnaire (\emph{map}) avec les \'el\'ements
    du flux regroupés par cl\'e.
   \\
\hline
	\begin{tabular}{@{}l@{}}
	\tt template<In, K=In, V=In, MapType> \\
	\tt reduceByKey(Reducer<In, V> r, Func<K*(In*)> fk)
	\end{tabular} &
	\TT{MapType} &
    Effectue une r\'eduction sur les valeurs de chaque cl\'e à l'aide d'un \TT{Reducer}. Voir la notion de \TT{Reducer}, Section.~\ref{reducer.sect}.
    \\
\hline
	\begin{tabular}{@{}l@{}}
	\tt template<T> \\
	\tt sort(Func<bool(T, T)> const\& compare)\label{sort.page}
	\end{tabular} &
	\TT{Collection<T, Container>} &
	Effectue le tri des \'el\'ements du flux selon l'ordre sp\'ecifi\'e par \TT{compare}. Note: Le premier \'el\'ement du flux de sortie n'est \'emis \emph{qu'apr\`es que la fin de flux ait \'et\'e rencontr\'ee}.
    \\
\hline
    \multicolumn{3}{c}{\ }
\\
    \multicolumn{3}{c}{\ }
\\
    \multicolumn{3}{c}{\ }
\\
	\multicolumn{3}{|c|}{\textbf{Execution} --- M\'ethode pour l'ex\'ecution parall\`ele du flux\label{concurrent.page}}
    \\      
\hline
	\begin{tabular}{@{}l@{}}
	\tt parallel(int workers = 1)
	\end{tabular} &
	\TT{Flow\&} &
	Sp\'ecifie le nombre de travailleurs \`a utiliser pour traiter les \'el\'ements du flux.
    \\                     
\hline    
\end{longtable}
\normalsize
\end{center}

\newpage
\KOMAoptions{paper=portrait,pagesize}
\recalctypearea
