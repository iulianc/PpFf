
\chapter{Spécifications semi-formelles du type {Reducer} et de la
méthode {groupByKey}}

\label{specification-reducer.ann}

Les descriptions (semi-)formelles pr\'esent\'ees dans cette annexe ont \'et\'e formul\'ees par mon directeur de recherche, le Prof.\ Guy Tremblay, et elles sont pr\'esent\'ees par souci de compl\'etude.

%%%%%%%%%%%%%%%%%%%%%%%%%%%%%%%%%%%%%%%%%%%%%%%%%%%%%%%%%%%%%%%%%%%%%%
\section{Description (semi-)formelle d'un \TT{Reducer}}
%%%%%%%%%%%%%%%%%%%%%%%%%%%%%%%%%%%%%%%%%%%%%%%%%%%%%%%%%%%%%%%%%%%%%%
\label{DescriptionFormelleReducer.ann}


Soit les \'el\'ements suivants~: 

\begin{itemize}

\item $T$ et $R$ des types de donn\'ees;

\item $S = [s_0, s_1, \ldots, s_k]$, un flux de donn\'ees, o\`u $s_i
\in T~(i=0, \ldots, k)$;

Soit $S_0, \ldots, S_n$ une partition de $S$;

\item Soit $v_0\in R$;

\item Soit $acc: R\times T \rightarrow R$;

\item Soit $comb: R\times R \rightarrow R$;


\end{itemize}

Soit $red$ un \TT{Reducer} avec les attributs $v_0, acc, comb$.

Alors $S$\TT{.reduce($red$)}


\begin{lstlisting}[language=haskell,escapechar=\!]
S.reduce( Reducer v0 acc comb )
  = foldr comb r!${}_0$! [r!${}_1$!, ..., r!${}_{k-1}$!]
    where
      [S!${}_0$!, S!${}_1$!, ..., S!${}_{k-1}$!] = splitIntoSubstreams S k
      [r!${}_0$!, r!${}_1$!, ..., r!${}_{k-1}$!] = [foldr acc v0 S!${}_i$! | i <- [0, !$\ldots$!, k-1]]


foldr f a []     = a
foldr f a (x:xs) = f x (foldr f a xs)
\end{lstlisting}



%%%%%%%%%%%%%%%%%%%%%%%%%%%%%%%%%%%%%%%%%%%%%%%%%%%%%%%%%%%%%%%%%%%%%%
\section{Description (semi-)formelle de \TT{GroupByKey}}
%%%%%%%%%%%%%%%%%%%%%%%%%%%%%%%%%%%%%%%%%%%%%%%%%%%%%%%%%%%%%%%%%%%%%%
\label{DescriptionFormelleGroupByKey.ann}


D\'enotons comme suit un \emph{map} $M$ qui associe la cl\'e $c_i$ \`a
la valeur $v_i$, pour $i=0, \ldots, n$~:%
%
\footnote{Dans du code \TT{C++}, un tel \emph{map} serait repr\'esent\'e
comme suit~: $\{\{c_0, v_0\}, \ldots, \{c_n, v_n\}\}$}
%
\begin{itemize}
\item $M = \{ c_i \mapsto v_i~\vert~0 \leq i \leq n \}$
\end{itemize}

Soit alors les \'el\'ements suivants~: 
\begin{itemize}
\item $T$, un type de donn\'ees;

\item $S = [s_0, s_1, \ldots, s_k]$, un flux de donn\'ees, o\`u $s_i
\in T~(i=0, \ldots, k)$;

\item $fc: T \rightarrow C$, une fonction sur  $T$ qui produit une
<<cl\'e>> de type $C$;

\item $fv: T \rightarrow V$, une fonction sur $T$ qui retourne une
<<valeur>> de type $V$;


\end{itemize}


Le r\'esultat d'un appel pour le flux $S$ de la m\'ethode
\TT{groupByKey} avec deux arguments peut alors \^etre d\'ecrit comme
suit~:
%
\begin{itemize}
\item $S.\TT{groupByKey}(fc, fv) = \{ c \mapsto vals(c, S)~\vert~c \in
cles(S)\}$

\item[] O\`u
\begin{itemize}
\item $cles(S) = \{ fc(s_i)~\vert~s_i \in S \}$
\item $vals(c, S) = \{ fv(s_i)~\vert~ s_i\in S \wedge fc(s_i) = c\}$
\end{itemize}

\end{itemize}


\bigskip


Quant \`a la m\'ethode $\TT{groupByKey}$ avec un seul argument, elle est
\'equivalente \`a celle avec deux arguments, mais o\`u le deuxi\`eme est simplement la fonction $id$entit\'e:
%
\begin{itemize}
\item $S.\TT{groupByKey}(fc) = S.\TT{groupByKey}(fc, id)$
\begin{Items}
\item[] O\`u $id(x) = x$
\end{Items}
\end{itemize}

