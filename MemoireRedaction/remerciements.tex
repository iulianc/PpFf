
   \parskip=0pt
   \vspace*{0.1 truecm} 
   \begin{center}
    {\uppercase { REMERCIEMENTS }}\par
   \end{center}
   \nobreak \vspace*{1.10 truecm}
   \parskip=2ex


\GT{Je crois qu'un des évaluateurs n'avait pas aimé <<grand
accomplissement>>.  Il semblait trouver que ça faisait trop
<<pompeux>>~:( À modifier!}

\GT{Il est aussi d'usage de remercier le directeur de recherche ou
l'organisme subventionnaire lorsque du support financier a été
offert. J'ai ajouté une phrase à ce  sujet.}

Tout grand accomplissement est rarement r\'ealis\'e par une seule personne. Celui-ci ne fait pas exception. Je suis reconnaissant \`a tous ceux et celles qui m'ont aid\'e et soutenu pendant ces ann\'ees de travail.

Tout d'abord, je voudrais remercier mon directeur de recherche, le professeur Guy Tremblay. Sans son soutien et son aide, ce m\'emoire n'aurait pas pu \^etre termin\'e. Gr\^ace \`a sa riche exp\`erience, il m'a guid\'e dans mes recherches. Il m'a donn\'e de nombreuses id\'ees, suggestions et critiques constructives.
%
Je le remercie aussi pour son support financier au cours de l'hiver et de l'été 2020.

Je dois \'egalement un grand merci au professeur Marco Aldinucci, de l'Universit\'e de Turin, pour son soutien technique pour l'application de \TT{FastFlow}. 

Je souhaite aussi remercier les analystes et techniciens syst\`emes des laboratoires informatiques et le personnel administratif qui ont permis \`a mon m\'emoire de se d\'erouler sans obstacle technique ou administratif.

Enfin, je tiens \`a remercier \`a ma famille, qui a toujours \'et\'e l\`a, m'ayant aid\'e moralement \`a surmonter les difficult\'es que j'ai eues durant ces ann\'ees de recherche. 


