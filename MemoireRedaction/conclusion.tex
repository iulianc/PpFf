
\begin{conclusion}
\label{conclusion.chap}

Dans ce m\'emoire, nous avons pr\'esent\'e \TT{PpFf}, une API qui peut \^etre utilis\'ee pour cr\'eer des applications de traitement de flux de donn\'ees. Impl\'ement\'ee par l'interm\'ediaire de la biblioth\`eque \TT{FastFlow} et \TT{C++ 17}, l'\TT{API} exploite les fonctionnalit\'es modernes du \TT{C++} et les concepts de programmation g\'en\'erique.

Con\c cue au-dessus de la biblioth\`eque \TT{FastFlow}, \TT{PpFf} fournit une abstraction de programmation souple et expressive qui facilite le d\'eveloppement d'applications parall\`eles, masquant ainsi non seulement la complexit\'e des m\'ecanismes de traitement concurrent, mais aussi le mod\`ele et le traitement de donn\'ees utilis\'es. L'abstraction dans \TT{PpFf} vise \`a simplifier la mani\`ere dont une s\'equence de donn\'ees est visualis\'ee et trait\'ee, permettant ainsi de r\'eutiliser les m\^emes op\'erateurs dans diff\'erents contextes et de r\'eutiliser les m\^emes algorithmes sur diff\'erents mod\`eles de donn\'ees (par exemple \TT{vector}, \TT{list}, \TT{set} etc.). Ces aspects diff\'erencient notre \TT{API} d'autres \TT{API} qui exposent diff\'erents types de donn\'ees \`a utiliser dans la m\^eme interface, ce qui oblige le d\'eveloppeur \`a r\'e\'evaluer ses op\'erations lorsque le contexte change.

Avec une \TT{API} claire et simple, \TT{PpFf} a \'et\'e con\c cu dans un style fonctionnel. Il expose donc \`a l'utilisateur un ensemble de m\'ethodes qui peuvent \^etre cha\^in\'ees afin de transformer les donn\'ees dans un style purement fonctionnel \`a l'aide d'une s\'erie de transformations.


Les r\'esultats d'exp\'eriences montrent que \TT{PpFf} peut fournir des applications de traitement de flux de donn\'ees \`a haut d\'ebit. Les bonnes performances obtenues par \TT{PpFf} sont d\^ues en partie au fait qu'il b\'en\'eficie directement de la conception optimis\'ee de \TT{FastFlow} pour le traitement de flux.

En ex\'ecutants diff\'erents tests, \TT{PpFf} a d\'emontr\'e la facilit\'e avec laquelle son \TT{API} permet de r\'esoudre certains probl\`emes classiques de traitement de flux de donn\'ees. Sa conception permet d'utiliser de nombreux concepts de programmation fonctionnels, par exemple, les \TT{lambdas} ou l'\'evaluation paresseuse. Ces aspects font de \TT{PpFf} un outil puissant.

Dans le chapitre sur les exp\'eriences nous avons \'egalement montr\'e que, dans certains cas, notre \TT{API} pouvait obtenir de meilleurs temps d'ex\'ecution qu'un programme \TT{Java} \'equivalent, avec les \TT{Stream}s introduits en Java 8.0, l'un des outils de traitement de donn\'ees le plus performant de nos jours.

Avec \TT{PpFf}, nous croyons que \TT{C++} a une chance de rivaliser avec \TT{Java} qui est consid\'er\'e souvent comme plus convivial que \TT{C++}. 


\section*{\textbf{Perspectives}}

Les travaux pr\'esent\'es dans ce m\'emoire pourraient \^etre \'etendus dans plusieurs directions, dont certaines sont d\'ecrites ci-dessous.

\textbf{Ajouter de nouvelles m\'ethodes parall\`eles} Afin de construire des requ\^etes compl\`etes sur une s\'equence de donn\'ees, nous pourrions \'etendre \TT{PpFf} pour prendre en charge davantage de m\'ethodes parall\`eles de traitement de flux et de donn\'ees, par exemple, \TT{distinct}, \TT{findAny}, \TT{findFirst}, \TT{forEach}, \TT{of}, etc. Ceci donnerait \`a l'utilisateur une plus grande flexibilit\'e pour la manipulation de donn\'ees.

\textbf{Impl\'ementer un syst\`eme distribu\'e} La version actuelle de \TT{PpFf} est con\c cue pour fonctionner uniquement en m\'emoire partag\'ee. Afin d'utiliser l'\TT{API} dans un environnement de donn\'ees massives (\emph{Big Data}), il faudrait impl\'ementer un module qui g\`ere la distribution des flux. Le module devrait g\'erer les probl\`emes li\'es aux syst\`emes distribu\'es tels que la coordination des nœuds, la reprise apr\`es une panne, l'allocation des t\^aches, etc. Ce module serait relativement facile \`a impl\'ementer dans \TT{PpFf} gr\^ace \`a \TT{FastFlow}, qui prend d\'ej\`a en charge l'ex\'ecution sur des syst\`emes distribu\'es.


\end{conclusion}




