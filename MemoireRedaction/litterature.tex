
\chapter{Un autre chapitre}
\label{un-autre-chap.chap}

Blah blah

Et voici un exemple de code ou pseudocode formatté avec la commande
{\tt alltt} (petit bout de code)~:
{\small
\begin{alltt}
  PROCEDURE fib( n:\ Nat ):\ Nat
  DEBUT
    SI n == 0 || n == 1 ALORS
      RETOURNER 1
    SINON
      RETOURNER fib(n-1) + fib(n-2)
    FIN
  FIN
\end{alltt}
}

Et un exemple de code dans une figure flottante (code plus long)~:
Algorithme~\ref{a1.algo}. Et un autre exemple dans une autre forme~:
Pseudocode~\ref{a2.pseudo}.

\begin{algorithme}
{\samepage\small
\begin{alltt}
  PROCEDURE fib'( n:\ Nat, A:\ ARRAY[*] OF Nat ):\ Nat
  DEBUT
     SI A[n] \(\neq\) PAS\_DEFINI ALORS
       // \emph{L'appel {\tt fib(n)} a déjà été calculé.}
       RETOURNER A[n]
     FIN
     
     // \emph{Premier appel à} fib(n).
     SI n == 0 || n == 1 ALORS
       r <- 1
     SINON
       r1 <- fib'( n-1, A )  
       r2 <- fib'( n-2, A )  
       r  <- r1 + r2
     FIN
     A[n] <- r 
     RETOURNER r
  FIN
\end{alltt}
}
\caption{L\'egende pour l'algorithme.}
\label{a1.algo}
\end{algorithme}

\begin{pseudocode}
{\samepage\small
\begin{alltt}
  PROCEDURE fib'( n:\ Nat, A:\ ARRAY[*] OF Nat ):\ Nat
  DEBUT
     SI A[n] \(\neq\) PAS\_DEFINI ALORS
       // \emph{L'appel {\tt fib(n)} a déjà été calculé.}
       RETOURNER A[n]
     FIN
     
     // \emph{Premier appel à} fib(n).
     SI n == 0 || n == 1 ALORS
       r <- 1
     SINON
       r1 <- fib'( n-1, A )  
       r2 <- fib'( n-2, A )  
       r  <- r1 + r2
     FIN
     A[n] <- r 
     RETOURNER r
  FIN
\end{alltt}
}
\caption{L\'egende pour l'algorithme.}
\label{a2.pseudo}
\end{pseudocode}

%
% Pour ajouter une reference dans la bibliographie, sans avoir de 
% reference explicite dans le texte
\nocite{Lalonde06url}
