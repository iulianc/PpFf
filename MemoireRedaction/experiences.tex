
\chapter{\'Etudes de cas et exp\'erimentations}
\label{experiences.chap}

Ce chapitre pr\'esente une \'evaluation exp\'erimentale de \TT{PpFf} afin d'analyser les possibilit\'es d'utilisation, en termes de performances, par rapport \`a d'autres approches d'ex\'ecution parall\`ele. Chaque exp\'erience est ex\'ecut\'ee plusieurs fois pour obtenir un r\'esultat stable. Dans un environnement de traitement de flux, les donn\'ees sont g\'en\'eralement transform\'ees au fur et \`a mesure de leur progression dans le \TT{pipeline}. Dans les sections suivantes, nous pr\'esentons deux applications de \TT{PpFf}. Ces cas d'utilisation ont \'et\'e choisis non seulement pour montrer certaines fonctionnalit\'es de l'API, mais \'egalement pour leur pertinence dans des sc\'enarios r\'eels. La section~\ref{wordcount.sect} pr\'esente une application permettant de calculer la fr\'equence d'occurrence des mots dans un texte --- le <<\emph{Hello World!} des syst\`emes de traitement de donn\'ees en mode \emph{batch} --- alors que la section~\ref{stockprice.sect} pr\'esente une application permettant de calculer des statistiques sur les prix d'indices boursiers --- un exemple typique de traitement de flux en ligne.


\section{Word Count}
\label{wordcount.sect}

D\'ecrit dans la section.~\ref{descriptionWordCount.sect}, \TT{WordCount} est une application simple qui compte le nombre d'occurrences des divers mots dans un fichier texte. L'application prend en entr\'ee un fichier texte et apr\`es le traitement, sort un conteneur de type \TT{map<string, int>} où la valeur de la cl\'e repr\'esente un mot dans le fichier et la valeur du conteneur de type \TT{int} repr\'esente l'occurrence du mot dans le fichier. Les codes sources des applications \TT{WordCount} dans \TT{PpFf} et \TT{Java} sont list\'es dans l'appendice~\ref{sourceCodeWordCountPpFf.ann} et l'appendice~\ref{sourceCodeWordCountJava.ann} respectivement.



\section{Stock Market}
\label{stockprice.sect}
