
\chapter{Exp\'erimentations~: Comparaisons de \ppff\ avec Java et FastFlow}
\label{experiences.chap}

Ce chapitre pr\'esente une \'evaluation exp\'erimentale de la biblioth\`eque \TT{PpFf} afin de comparer ses performances avec d'autres approches d'ex\'ecution parall\`ele, plus sp\'ecifiquement avec \TT{Java} et \TT{FastFlow}.
%
Dans les sections~\ref{wordcount.sect} et~\ref{stockprice.sect}, nous pr\'esentons deux applications \'ecrites avec \PpFf. Ces applications ont \'et\'e choisies non seulement pour montrer certaines fonctionnalit\'es de notre biblioth\`eque, mais \'egalement pour leur pertinence dans des sc\'enarios typiques. La section~\ref{wordcount.sect} pr\'esente une application permettant de calculer le nombre d'occurrences des mots dans un texte --- le <<\emph{Hello World!}>> des syst\`emes de traitement de donn\'ees en mode \emph{batch} --- alors que la section~\ref{stockprice.sect} pr\'esente une application permettant de calculer des statistiques sur les prix d'indices boursiers --- un exemple typique de traitement de flux en ligne (\emph{online data processing}). Mais tout d'abord, nous pr\'esentons la fa\c{c}on dont nos exp\'erimentations ont \'et\'e effectu\'ees.

%Finalement, la section~\ref{coutsPpFf.sect} pr\'esente une application permettant de d\'eterminer les surco\^uts introduits par \TT{PpFf} par rapport \`a \TT{FastFlow} --- un \emph{micro benchmark} consistant en un pipeline avec un seul op\'erateur. 



\section{M\'ethode utilis\'ee pour les exp\'erimentations}
\label{usedMethodsForBenchmarks.chap}

\subsection{Caractéristiques des machines et compilateurs utilisés}

Chaque syst\`eme informatique a des caract\'eristiques propres. Quelques-uns des facteurs qui influencent les performances d'un tel syst\`eme sont le type de processeur, le nombre de processeurs ou de c\oe{}urs, et la vitesse des processeurs. 


\newcommand{\LARGEUR}{3cm}

\begin{table}
\begin{tabular}{|p{3cm}|p{\LARGEUR}|p{\LARGEUR}|p{\LARGEUR}|}
\hline
  & \M1 & \M2 & \M3
\\\hline
\textbf{OS} & CentOS 7.8.2003 & CentOS 7.8.2003 & CentOS 7.6.1810
\\\hline
\textbf{Architecture} &  x86\_64 & x86\_64 & x86\_64
\\\hline
\textbf{Type de processeur} & GenuineIntel  & AuthenticAMD & GenuineIntel
\\\hline
\textbf{Vitesse du processeur (GHz)} & 2.66 & 2.30 & 3.96
\\\hline
\textbf{Mono/multi-usager} & Multi & Multi & Mono
\\\hline
\textbf{Nb.~coeurs physiques} & 16 & 32 & 4
\\\hline
\textbf{Nb.~coeurs logiques} & 16 & 64 & 8
\\\hline
\texttt{java}
  & \texttt{openjdk 11.0.7 2020-04-14 LTS}
  & \texttt{java version "1.8.0\_51"}
  & \texttt{openjdk version "11.0.7" 2020-04-14 LTS}
\\\hline
\texttt{g++ (GCC)}
   & 8.3.1
   & 8.3.1 
   & 8.3.0
\\\hline
\end{tabular}
\caption[Les caract\'eristiques des machines utilis\'ees dans nos exp\'eriences.]{Les caract\'eristiques des machines utilis\'ees dans nos exp\'eriences. Le tableau d\'ecrit pour chaque machine le syst\`eme d'exploitation utilis\'e, le type de la machine (mono- ou multi-usager), le nombre de coeurs (physiques vs.\ logiques) et les versions des compilateurs utilis\'es dans nos exp\'eriences.}
\label{machines.table}
\end{table}


Afin d'avoir des r\'esultats plus représentatifs, nous avons conduit nos exp\'eriences sur trois machines diff\'erentes. Le tableau~\ref{machines.table} montre les caract\'eristiques de ces machines. Les machines \M1 et \M2 sont des machines multi-usagers, tandis que la machine \M3 est une machine mono-usager sur laquelle les exp\'eriences ont \'et\/e roul\'ees sans interf\'erence.

\subsection{Choix des programmes comparés}

Le fonctionnement des programmes \TT{Java}, \TT{PpFf} et~\TT{FastFlow} n'est pas le m\^eme. Alors que \TT{PpFf} et \TT{FastFlow} permettent de varier le nombre de fils d'ex\'ecution (\emph{threads}), \TT{Java} ne le permet pas. Afin de montrer les meilleurs temps d'ex\'ecution et l'\'evolutivit\'e de \TT{PpFf}, des exp\'eriences pr\'eliminaires ont \'et\'e effectu\'ees, sur chaque machine, pour identifier les meilleures versions, et ce tant pour \TT{PpFf} que pour \TT{Java} et \TT{FastFlow}.
Ces exp\'eriences pr\'eliminaires ont \'et\'e conduites avec un nombre de r\'ep\'etitions de 10 ou 20 et avec une quantit\'e <<moyenne>> ou <<grande>> de donn\'ees, selon les cas. 

Dans le cas de \TT{PpFf} et \TT{FastFlow}, l'objectif \'etait de d\'eterminer le meilleur niveau de parall\'elisme \`a utiliser dans les \emph{farm}s, c'est-\`a-dire, pour \TT{PpFf}, la valeur pour la m\'ethode \TT{parallel()}, pour le parall\'elisme de donn\'ees.
%
Par contre, dans le cas de \TT{Java}, l'objectif était de comparer diverses versions~: avec ou sans JIT, séquentielle ou parallèle, avec ou sans \emph{warmup} (voir ci-bas).
%
C'est la version parallèle avec \emph{warmup} et \emph{JIT} qui a \'et\'e utilis\'ee.%
%
\footnote{Initialement, des expériences préliminaires ont aussi été
effectuées en désactivant complètement le JIT. Toutefois, les temps
d'exécutions étaient alors tellement longs que cette variante a été
ignorée.}


L'effet de pr\'echauffage (\emph{warmup} en anglais) est g\'en\'eralement d\^u au chargement des classes et \`a l'interpr\'etation du \emph{bytecode} au d\'emarrage du programme plutôt qu'à l'exécution directe d'instructions machines. Lorsqu'une nouvelle application d\'emarre, toutes les classes requises sont charg\'ees en m\'emoire par le mod\`ele de chargement paresseux (\emph{lazy loading} en anglais). Un tel mod\`ele est couramment utilis\'e pour reporter l'initialisation d'un objet jusqu'au moment o\`u il est n\'ecessaire.
%
\label{jitDescription.sect}
%
Il y a aussi l'effet de la compilation 
\emph{JIT}~\citep{cramer1997compiling} (\emph{Just-In-Time compiler}),
%
un composant de l'environnement d'ex\'ecution Java qui am\'eliore les performances des applications en compilant le \emph{bytecode} de la machine virtuelle en code machine \emph{au moment de l'ex\'ecution}. Le \emph{bytecode} est l'ensemble des instructions de la \emph{JVM} (\emph{Java Virtual Machine}) qui permet aux applications d'\^etre ex\'ecut\'ees sur plusieurs plates-formes. La conversion du \emph{bytecode} en langage machine a un impact positif significatif sur la vitesse d'ex\'ecution une fois que le code machine s'exécute directement.

Dans toutes nos exp\'eriences, les programmes \TT{Java} comparés avec \ppff\ ont donc \'et\'e pr\'echauff\'es en lan\c{c}ant une proc\'edure de pr\'echauffage avant de mesurer le temps pour la portion de code pertinente. Cette proc\'edure de pr\'echauffage a consisté à faire un traitement préalable d'une \emph{fraction} des données, traitement utilisant toutes les m\'ethodes utilis\'ees par la suite.

\subsection{Temps mesurés}
 
Il faut noter que le temps mesur\'e dans toutes les exp\'eriences \emph{exclut le lancement du programme}. La mesure du temps se fait de l'int\'erieur du programme m\^eme, une fois celui-ci lanc\'e. Lorsque le traitement est termin\'e, le temps est \'emis en sortie du programme sur \TT{stdout}. Donc, le temps n'est pas mesur\'e avec la commande <<\TT{time}>>. Notamment, dans le cas de \TT{Java}, le temps mesuré exclut donc le temps de lancement de la machine virtuelle. De plus, les quantit\'es de donn\'ees utilis\'ees dans les exp\'eriences ont \'et\'e choisies de sorte que les temps d'ex\'ecution soient d'au moins 200 millisecondes.


\subsection{Exemples d'expériences préliminaires}

\graphe{WordCount-temps-java-11-20}{Temps Java}

\graphe{WordCount-temps-java-2-10}{Temps PpFf}
%\graphe{WordCount-log-temps-java-2-10}{Temps PpFf (log)}

Afin de montrer les \'etapes des diverses exp\'eriences qui ont conduit aux r\'esultats finaux, l'annexe~\ref{ExperiencesPreliminairesWordCount.ann} pr\'esente un extrait d'un fichier de configuration \TT{WordCount-bm-config.rb}, utilis\'e pour l'ex\'ecution des expériences, alors que les figures~\grapheref{WordCount-temps-java-11-20} et~\grapheref{WordCount-temps-java-2-10} présentent quelques-uns des r\'esultats préliminaires obtenus.

Cr\'e\'e par mon directeur de recherche, ce script permet de configurer les expériences \`a ex\'ecuter. Dans ce fichier on peut sp\'ecifier tous les param\`etres dont on a besoin : la machine pour laquelle est définie l'expérience, les quantités de donn\'ees, le nombre de r\'ep\'etitions, et les programmes \`a ex\'ecuter. Les donn\'ees sont regroup\'ees par taille en diverses cat\'egories, les plus importantes étant \TT{donnees\_preliminaires}, \TT{pas\_mal\_de\_donnees} et \TT{beaucoup\_de\_donnees}. Les deux premi\`eres cat\'egories ont servi pour d\'eterminer les meilleures versions \`a utiliser pour chacun des programmes. 

Par exemple, les graphes des figures~\grapheref{WordCount-temps-java-11-20} et~\grapheref{WordCount-temps-java-2-10} montrent les temps d'ex\'ecution pour \TT{WordCount} sur la machine \M1 pour le programme \TT{WordCount.java} (expérience no.~11) et pour le programme \TT{WordCount.cpp}, donc version \TT{PpFf} (expériences no.~2). Les temps sont en millisecondes (ms), obtenus en prenant la moyenne de 20 ou 10 ex\'ecutions, selon le cas.
%
Pour la version \TT{Java}, quatre séries de mesures ont été
effectuées~: séquentielle sans préchauffage (\TT{Java-}), séquentielle
avec préchauffage (\TT{Java}), parallèle sans préchauffage
(\TT{Java+}), et parallèle avec préchauffage (\TT{Java*}) --- pour les résultats, voir l'annexe~\ref{wordcount-java.ann}.
%
Pour la version \ppff, trois  s\'eries de mesures ont \'et\'e effectu\'ees : \TT{PpFf-1} avec une seule instance parallèle d'un \emph{farm}, \TT{PpFf-2} avec deux et \TT{PpFf-3} avec trois.



L'objectif de ces mesures pr\'eliminaires est de d\'eterminer la valeur qui semble la meilleure pour les temps d'ex\'ecution. On peut observer que le temps d'ex\'ecution pour \TT{PpFf-3} est beaucoup plus grand que les deux autres (figure~\grapheref{WordCount-temps-java-2-10}).
%
Pour mieux distinguer les performances entre deux ou plusieurs versions, une échelle logarithmique peut aussi être utilis\'ee. 
%
Pour la machine \M1, \TT{PpFf-2} est donc la version qui sera compar\'ee aux autres programmes lors des expériences finales. Ces expériences comparent donc les meilleures versions entre elles, en utilisant de plus grandes quantit\'es de donn\'ees et avec un plus grand nombre de r\'ep\'etitions, soit 40.


Le nombre de r\'ep\'etitions indique combien de fois chaque programme est exécuté et son temps mesuré.
%
On calcule ensuite la moyenne et l'écart-type. Dans un graphe comme celui de la figure~\grapheref{WordCount-temps-java-11-20}, les moyennes sont repr\'esent\'ees par les valeurs qui composent la courbe sur le graphe et les dispersions par des petits barres verticales (par ex., voir les valeurs pour \TT{Java+}
de la figure~\grapheref{WordCount-temps-java-11-20}). Plus précisément, la barre verticale indique un intervalle de 2 écart-types, donc un intervalle qui contient approximativement \emph{95~\% des temps mesurés}.


\GT{Ci-bas. Je crois qu'il serait aussi bon d'inclure \TT{Java} = version séquentielle avec warm-up. À voir!!}

Chaque s\'erie d'exp\'eriences finales inclut le programme \TT{Seq}, une version s\'equentielle utilisant les m\^emes fonctions auxiliaires que les versions pour \TT{PpFf} et \TT{FastFlow}, mais s'ex\'ecutant de fa\c con s\'equentielle. Ce programme a aussi \'et\'e utilis\'e pour d\'eterminer les acc\'el\'erations. Le concept d'acc\'el\'eration d\'etermine \`a quel point un programme parall\`ele est plus rapide qu'un programme s\'equentiel \'equivalent. On distingue deux types d'acc\'elérations : relative ou absolue. Dans nos mesures, nous avons utilis\'e l'acc\'el\'eration \emph{absolue}. L'acc\'el\'eration absolue compare le programme ex\'ecut\'e sur une machine multiprocesseurs avec le meilleur programme s\'equentiel qui r\'esout le m\^eme probl\`eme. 

Afin d'illustrer aussi la dispersion des acc\'el\'erations, l'accélération moyenne a été calculée, ainsi deux autres valeurs donnant un intervalle pour la valeur minimale et maximale de l'acc\'el\'eration. Repr\'esent\'ees aussi par des petites barres verticales dans les graphes des sections~\ref{wordcount.sect} et~\ref{stockprice.sect}, ces valeurs ainsi que l'accélération moyenne sont calcul\'ees comme suit~: 

\begin{itemize}
\item acc. moyenne  =  temps moyen séq. / temps moyen par.
\item acc. min  =  temps min séq. / temps max par.
\item acc. max = temps max séq. / temps min par.
\end{itemize}

\section{Analyse de l'application \TT{WordCount}}
\label{wordcount.sect}



\subsection{Description de l'application \TT{WordCount}}

Dans la section~\ref{descriptionWordCount.sect}, nous avons pr\'esent\'e \TT{WordCount}, une application simple qui compte le nombre d'occurrences des divers mots dans un fichier texte. L'application prend en entr\'ee un fichier texte et produit un conteneur de type \TT{map<string, int>} où la cl\'e repr\'esente un mot dans le fichier et la valeur  type \TT{int}  associ\'ee repr\'esente le nombre d'occurrences du mot dans le fichier. Des extraits des programmes utilis\'es pour les exp\'erimentations pour \TT{WordCount} en~\TT{Java}, \TT{C++} version \TT{Seq}uentielle, \TT{PpFf} et \TT{FastFlow} sont pr\'esent\'es dans l'annexe~\ref{appendice-code-experiences.ann}.

\subsection{Mesures obtenues et analyse des r\'esultats}



\begin{figure}
\grapheH{WordCount-temps-java-3001-40}

\grapheH{WordCount-temps-japet-3002-40}

\grapheH{WordCount-temps-c34581-3003-40}

\caption[Les temps d'exécution des programmes pour \TT{WordCount} sur
les machines \M1, \M2 et \M3.]{Les temps d'exécution des programmes
pour \TT{WordCount} sur les machines \M1, \M2 et \M3. L'axe des $x$
indique le nombre de mots traités. L'axe des $y$ indique le temps
d'exécution, en millisecondes.}
\label{WordCount-temps.fig}
\end{figure}


\begin{figure}
\grapheH{WordCount-debits-java-3001-40}

\grapheH{WordCount-debits-japet-3002-40}

\grapheH{WordCount-debits-c34581-3003-40}

\caption[Les débits pour \TT{WordCount} sur
les machines \M1, \M2 et \M3.]{Les débits des programmes
pour \TT{WordCount} sur les machines \M1, \M2 et \M3. L'axe des $x$
indique le nombre de mots traités. L'axe des $y$ indique le nombre de
milliers de mots par seconde (K-mots/s).}
\label{WordCount-debits.fig}
\end{figure}


\begin{figure}
\grapheH{WordCount-accs-java-3001-40}

\grapheH{WordCount-accs-japet-3002-40}

\grapheH{WordCount-accs-c34581-3003-40}

\caption[Les accélérations pour \TT{WordCount} sur les machines \M1,
\M2 et \M3.]{Les accélérations des programmes pour \TT{WordCount} sur
les machines \M1, \M2 et \M3. L'axe des $x$ indique le nombre de mots
traités. L'axe des $y$ indique l'accélération absolue par rapport à
\TT{WordCountSeq.cpp} (\TT{Seq}).}
\label{WordCount-accs.fig}
\end{figure}


Dans cette section, nous \'evaluons l'application \TT{WordCount} en examinant le temps d'ex\'ecution, le d\'edit et l'acc\'el\'eration sur les trois machines : \M1, \M2 et \M3. Tel que d\'ecrit dans la section~\ref{usedMethodsForBenchmarks.chap}, des exp\'eriences pr\'eliminaires ont \'et\'e effectu\'ees afin de choisir les meilleures versions. Dans le cas de \TT{Java}, la meilleure version choisie est celle avec \emph{warmup} et \emph{JIT}. Elle est indiquée dans chaque graphe avec la notation \TT{Java*}. Dans le cas de \TT{PpFf} et \TT{FastFlow}, les exp\'eriences ont \'et\'e men\'ees en variant les nombres d'instances parall\`eles d'un \emph{farm}. Deux instances parall\`eles d'un \emph{farm} ont \'et\'e utilis\'es sur la machine \M1, huit sur la machine \M2 et seulement une sur la machine \M3. Le suffixe entier dans les indicateurs pour \TT{PpFf} et \TT{FastFlow} dans chaque graphe repr\'esente donc ce nombre d'instances parall\`eles d'un \emph{farm}, par exemple \TT{PpFf-2} utilise deux instances parall\`eles d'un \emph{farm}. Chaque exp\'erience inclut aussi le programme s\'equentiel, indiqué sur chaque graphe par \TT{Seq}, utilisant  les m\^emes fonctions auxiliaires que \TT{PpFf} et \TT{FastFlow}.

Les valeurs pour les unit\'es de mesures -- le temps d'ex\'ecution, le d\'ebit et l'acc\'el\'eration qui ont servi comme r\'ef\'erence pour comparer les trois programmes -- sont indiqu\'ees sur l'axe des~$y$ de chaque graphe, alors que le nombre de mots trait\'es est indiqu\'e sur l'axe des~$x$. Afin de conna\^itre l'impact de la quantit\'e de donn\'ees \`a traiter, les exp\'eriences ont \'et\'e men\'ees en utilisant plusieurs ensembles de donn\'ees. Chaque ensemble de donn\'ees --— un fichier sur disque --— contient un nombre croissant de mots. Ces nombres de mots varient de 752~856 \`a 10~185~035. Les r\'esultats finaux sont des moyennes pour 40 r\'ep\'etitions. Ils sont pr\'esent\'es comme suit : 



\begin{itemize}

\item La figure~\ref{WordCount-temps.fig} pr\'esente les temps
d'ex\'ecution sur les machines \M1, \M2 et \M3.

\item La figure~\ref{WordCount-debits.fig} pr\'esente les débits sur
les machines \M1, \M2 et \M3.

\item La figure~\ref{WordCount-accs.fig} pr\'esente les accélérations
par rapport à la version \TT{Seq}entielle
sur les machines \M1, \M2 et \M3.
\end{itemize}


Il faut pr\'eciser que pour \TT{WordCount}, le r\'esultat n'est pas tri\'e. Du point de vue de la parall\'elisation, le tri est plut\^ot lie \'a l'algorithme de tri et non \`a la parall\'elisation. Pour toutes les versions, c'est un \emph{dictionnaire} qui est produit --- \TT{unordered\_map} en \TT{C++}, \TT{HashMap} en \TT{Java} --- et la mesure du temps d'exécution se termine une fois le \emph{dictionnaire} construit.

En comparant les temps d'ex\'ecution entre \TT{PpFf} et \TT{Java}, on constate que, pour les machines \M1 et \M3, \TT{Java} est plus performant que \TT{PpFf}. La gestion de \emph{threads} entre les deux programmes diff\`ere. \TT{PpFf} ex\'ecute chaque op\'eration d'un \TT{pipeline} sur un \emph{thread} diff\'erent. Par exemple, les cinq op\'erations de \TT{WordCount} --- \TT{source}, \TT{flatMap}, \TT{map}, \TT{find} et \TT{reduceByKey} --- s'ex\'ecutent sur cinq \emph{threads} diff\'erents. Par contre, \TT{Java} g\`ere la cr\'eation de \emph{threads} par l'interm\'ediaire du \emph{framework} \TT{fork/join}. D\'ecrit dans le chapitre~\ref{outils_connus.chap}, le \emph{framework} divise une t\^ache en plus petites sous-t\^aches ind\'ependantes, et ce r\'ecursivement jusqu'\`a ce qu'elles soient assez simples pour \^etre ex\'ecut\'ees. Ce m\'ecanisme permet \`a \TT{Java} d'\^etre plus efficace que \TT{PpFf--1} sur les machines \M1 et \M3. \TT{PpFf} est plus rapide que \TT{Java} sur la machine \M2. La figure~\ref{WordCount-debits.fig}, machine \M2, montre cet aspect. Par rapport aux machines \M1 et \M3, \M2 dispose de plusieurs processeurs. Cela a permis d'augmenter le nombre d'instances parall\`eles d'un \emph{farm} dans le programme \TT{PpFf} en arrivant \`a huit et en cons\'equence \TT{PpFf} est plus performant que \TT{Java}. 


En comparant les temps d'ex\'ecution entre les programmes \TT{PpFf} et \TT{FastFlow}, on constate que, \TT{FastFlow} est l\'eg\`erement plus performant que \TT{PpFf}. On rappelle que \TT{PpFf} est impl\'ement\'e au–dessus de la biblioth\`eque \TT{FastFlow}. C'est-\`a-dire que \TT{PpFf} introduit un faible surco\^ut par rapport \`a \TT{FastFlow}. Pourtant, les surco\^uts introduits par \TT{PpFf} par rapport \`a \TT{FastFlow} restent faibles --— c'est ce qui explique, surtout dans les figures~\ref{WordCount-temps.fig} et~\ref{WordCount-debits.fig}, que les lignes du graphe pour \TT{PfFf} et \TT{FastFlow}, tant pour le temp d'ex\'ecution que pour le d\'ebit, sont quasiment identiques. La plus grande diff\'erence entre les temps d'ex\'ecution de deux programmes peut \^etre constat\'ee sur le machin \M3 (voir la figure~\ref{WordCount-temps.fig}). Or, en prenant en consid\'eration le grand volume de traitement des deux applications, cette diff\'erence semble n\'egligeable.

Afin de mieux comparer les deux programmes, \TT{PpFf} et \TT{Java}, nous avons aussi calculé, à partir des mêmes séries d'exp\'eriences, le d\'ebit, soit le nombre de mots trait\'e par seconde. Tel que montr\'e dans la figure~\ref{WordCount-debits.fig}, ces débits ont \'et\'e calculés pour les trois machines. L'axe des $x$ indique le nombre de mots trait\'es et l'axe des $y$ indique le nombre de milliers de mots par seconde (\TT{K-mots/s}). Un point int\'eressant, qui peut \^etre observ\'e dans les diagrammes de d\'ebits, est que le d\'ebit est relativement constant. En prenant comme exemple le diagramme pour \TT{PpFf-2} pour \M1 de la figure~\ref{WordCount-debits.fig}, on peut noter que le d\'ebit reste relativement stable peu importe la taille du fichier. Cela d\'emontre que \TT{PpFf} est efficace non seulement pour un petit volume de travail, mais aussi pour de grands traitements de donn\'ees.


La derni\`ere s\'erie de résultats tirée de nos exp\'eriences vise \`a comparer les acc\'el\'erations. Illustr\'ees sur la figure~\ref{WordCount-accs.fig}, les acc\'el\'erations indiquent \`a quel point un programme parall\`ele est plus rapide qu'un programme s\'equentiel \'equivalent. En comparant les acc\'el\'erations du programme \TT{PpFf} avec celle de \TT{FastFlow}, on peut observer qu'elles sont presque identiques, et ce pour les trois machines. Cela d\'emontre que les surco\^uts introduits par \TT{PpFf} par rapport \`a \TT{FastFlow} sont n\'egligeables. Par contre, \TT{Java} a de meilleures acc\'el\'erations pour \M1 et \M3, mais pas pour \M2. Comme on peut le constater dans tableau~\ref{machines.table}, la machine \M2 dispose d'un plus grand nombre de processeurs par rapport \`a \M1 et \M3. Cet avantage est exploit\'e par les instances de \emph{farm}, permettant \`a \TT{PpFf} d'obtenir une meilleure acc\'el\'eration que \TT{Java} sur cette machine.

\GT{il faudrait quand même écrire quelques mots sur le
fait que Java a de meilleures accélérations pour \M1 et \M3, mais pas
pour \M2. Cause/explication possible: \M2 a un plus grand nombre de
processeurs, qui peuvent être exploités par les instances de
\emph{farm}?}

\IC{J'ai ajouté votre commentaire.}


\begin{figure}
\grapheH{WordCount-accsrel-java-3001-40-1234}

\grapheH{WordCount-accsrel-japet-3002-40-1234}

\grapheH{WordCount-accsrel-c34581-3003-40-1234}

\caption[Les accélérations \emph{relatives} pour \TT{WordCount} sur les machines \M1,
\M2 et \M3.]{Les accélérations relatives des programmes pour \TT{WordCount} sur
les machines \M1, \M2 et \M3 par rapport à leur version séquentielle. L'axe des $x$ indique le nombre de mots
traités. L'axe des $y$ indique l'accélération absolue par rapport à
\TT{WordCountSeq.cpp} (\TT{Seq}). Les résultats pour la version
\TT{FastFlow} ont été omis (semblables à ceux de la version \ppff).}
\label{WordCount-accsrel.fig}
\end{figure}


\GT{J'ai ajouté la figure avec les accélérations relatives. Il faudra
voir si ça vaut la peine de l'expliquer et d'en discuter! Mais je
voulais voir ce dont elle aurait l'air. }



\section{Analyse de l'application \TT{StockPrice}}
\label{stockprice.sect}

Dans le monde informatique actuel, les institutions financi\`eres produisent d'\'enormes quantit\'es d'informations, par ex., des informations sur les march\'es boursiers. Un probl\`eme important qu'elles rencontrent consiste \`a trouver des moyens efficaces pour r\'esumer et visualiser les donn\'ees afin de produire des informations utiles sur le comportement du march\'e, notamment pour prendre des d\'ecisions d'investissement. Cette section pr\'esente une application qui calcule le prix maximum pour diverses actions d'un marché boursier. Des extraits des programmes utilis\'es pour les exp\'erimentations pour \TT{StockPrice} en~\TT{Java}, \TT{C++} version \TT{Seq}uentielle, \TT{PpFf} et \TT{FastFlow} sont pr\'esent\'es dans l'annexe~\ref{appendice-code-stockprice.ann}.


\subsection{Description de l'application}

L'application \TT{StockPrice} calcule le prix d'une action en utilisant le modèle \emph{Black-Scholes}~\citep{macbeth1979empirical}. Ce mod\`ele d'\'evaluation est utilis\'e pour d\'eterminer le prix juste ou la valeur th\'eorique d'une option d'achat ou de vente, et ce en fonction de six variables telles que la valeur de l'action sous-jacente, le prix d'exercice, le taux d'int\'er\^et sans risque, la volatilit\'e du prix de l'action, la dur\'ee et le type d'option. 

\begin{lstlisting}[float,label={StockPrice-code.listing},gobble=4,basicstyle=\ttfamily\footnotesize,language=c++,caption={Extrait du code de \TT{StockPrice.cpp} (version \ppff).},frame=single]
    Reducer<StockAndPrice, double>
       reducer(0.0, 
               [](double maxPrice, StockAndPrice sp) {
                  return std::max(maxPrice, sp.StockPrice);
               },
               [](double max, double workerResult) { 
                  return std::max(max, workerResult);
               });
    std::unordered_map<std::string, double> currentResult =
        Flow
        ::source(inputFile)
        .parallel(nbFarmWorkers)
        .map<std::string, OptionData>(getOptionData)
        .map<OptionData, StockAndPrice>(calculateStockPrice)
        .reduceByKey<StockAndPrice, std::string, double>(
               reducer, 
               [](StockAndPrice* sp) { return &(sp->StockName); } );
\end{lstlisting}


L'application \TT{StockPrice} est compos\'ee de cinq op\'erations principales, comme on peut le voir dans le listing~\ref{StockPrice-code.listing} qui présente un extrait de la version \ppff.

\begin{lstlisting}[
label={exampleInfoActionFromFile},
language=c++,
caption={Un exemple illustrant l'information sur des actions contenues dans le fichier.},
frame=single,
float]
SNY 100.00 90.00 0.1000 0.00 0.10 1.00 C 0.00 18.6308591206674982
JCI 100.00 100.00 0.1000 0.00 0.10 0.50 C 0.00 5.8502736042849798
DSX 100.00 100.00 0.1000 0.00 0.10 1.00 C 0.00 10.3081472436668
LILA 100.00 110.00 0.1000 0.00 0.10 0.10 C 0.00 0.003523074865
NVS 100.00 110.00 0.1000 0.00 0.10 0.50 C 0.00 1.1407228438274099
FLML 100.00 110.00 0.1000 0.00 0.10 1.00 C 0.00 4.216747020308
TEF 100.00 90.00 0.1000 0.00 0.25 0.10 C 0.00 11.1352446183467002
DXB 100.00 90.00 0.1000 0.00 0.25 0.50 C 0.00 16.0926388440922991
HSEA 100.00 90.00 0.1000 0.00 0.25 1.00 C 0.00 21.16345465848
LENS 100.00 100.00 0.1000 0.00 0.25 0.10 C 0.00 3.65996266031
\end{lstlisting}


\GT{Comme tu décrivais ce que faisait le code, il m'a semblé
préférable qu'il soit visible, juste à coté.}


\begin{itemize}

\item Une op\'eration \TT{Flow::source} qui d\'efinit la source du flux de donn\'ees. Ici, la source est constitu\'ee par les lignes contenues dans un fichier. Le listing~\ref{exampleInfoActionFromFile} montre un exemple avec quelques enregistrements tir\'es d'un des fichiers de données. 

Un enregistrement est identifi\'e par les informations suivantes : le nom de l'action, la valeur actuelle de l'action sous-jacente, le prix d'exercice, le taux d'int\'er\^et sans risque, le taux de dividende, la volatilit\'e du prix de l'action, le temps qu'il reste \`a l'option avant son \'ech\'eance (exprim\'e en ann\'ees), le type d'option (\TT{C=CALL}~: prix pour une option d'achat~; \TT{P=PUT}~: prix pour une option de vente), la valeur de dividende et la valeur de r\'ef\'erence \TT{DerivaGem}. 
Les valeurs \TT{DerivaGem}, la valeur et le taux de dividende ne sont pas utilis\'es dans \TT{StockPrice} pour calculer le prix d'une action.

\item Une op\'eration \TT{parallel} qui r\'epartit les \'el\'ements du flux entre divers \emph{threads} de parallélisme de données.
Toutes les \'etapes qui suivent cette op\'eration seront donc ex\'ecut\'ees en parall\`ele, via une \emph{farm}.

\item Une op\'eration \TT{map}, qui permet d'extraire le nom et les options de chaque action.

\item  Une autre op\'eration \TT{map} qui calcule le prix de chaque action. L'algorithme utilis\'e est celui de \emph{Black-Scholes}~\citep{macbeth1979empirical}. 

\item Une derni\`ere op\'eration, \TT{reduceByKey}, qui extrait le prix maximum pour chaque action.

\end{itemize}

\subsection{Mesures obtenues et analyse des r\'esultats}

\begin{figure}
\grapheH{StockPrice-temps-java-1001-40}

\grapheH{StockPrice-temps-japet-1002-40}

\grapheH{StockPrice-temps-c34581-1003-40}

\caption[Les temps d'exécution des programmes pour \TT{StockPrice} sur
les machines \M1, \M2 et \M3.]{Les temps d'exécution des programmes
pour \TT{StockPrice} sur les machines \M1, \M2 et \M3. L'axe des $x$
indique le nombre de transactions traitées. L'axe des $y$ indique le temps d'exécution, en millisecondes.}
\label{StockPrice-temps.fig}
\end{figure}


\begin{figure}
\grapheH{StockPrice-debits-java-1001-40}

\grapheH{StockPrice-debits-japet-1002-40}

\grapheH{StockPrice-debits-c34581-1003-40}

\caption[Les débits pour \TT{StockPrice} sur
les machines \M1, \M2 et \M3.]{Les débits des programmes
pour \TT{StockPrice} sur les machines \M1, \M2 et \M3. L'axe des $x$
indique le nombre de transactions traitées. L'axe des $y$ indique le nombre de milliers de transactions par seconde (K-transactions/s).}
\label{StockPrice-debits.fig}
\end{figure}


Dans cette section, nous \'evaluons l'application \TT{StockPrice} en examinant le temps d'ex\'ecution et le d\'edit sur les trois machines : \M1, \M2 et \M3. Comme dans le cas de \TT{WordCount}, des exp\'eriences pr\'eliminaires ont \'et\'e effectu\'ees afin de choisir les meilleures versions. Dans le cas de \TT{Java}, la meilleure version choisie est celle avec \emph{warmup} et \emph{JIT}. Elle est indiqu\'ee dans chaque graphe avec la notation \TT{Java*}. Dans le cas de \TT{PpFf} et \TT{FastFlow}, les exp\'eriences ont \'et\'e men\'ees en variant les nombres d'instances parall\`eles d'un \emph{farm}. Les meilleures versions choisies pour les deux programmes sont les versions utilisant quatre instances parall\`eles d'un \emph{farm} sur les machines \M1 et \M2 et deux sur la machine \M3. Le suffixe entier dans les indicateurs pour \TT{PpFf} et \TT{FastFlow} dans chaque graphe repr\'esente donc ce nombre d'instances parall\`eles d'un \emph{farm}. Chaque exp\'erience inclut aussi le programme s\'equentiel, indiqu\'e sur chaque graphe par \TT{Seq}. Chaque programme calcule le prix maximum pour plusieurs actions. Les temps d'ex\'ecution et les d\'ebit r\'esultants sont repr\'esent\'es sur l'axe des~$y$ de chaque graphe, alors que les nombres de transactions trait\'ees sont repr\'esent\'es sur l'axe des~$x$. Les r\'esultats de exp\'eriences finaux sont des moyennes pour 40 r\'ep\'etitions. Ils sont pr\'esent\'es comme suit :


\begin{itemize}

\item La figure~\ref{StockPrice-temps.fig} pr\'esente les temps d'ex\'ecution sur les machines \M1, \M2 et \M3.

\item La figure~\ref{StockPrice-debits.fig} pr\'esente les d\'ebits sur les machines \M1, \M2 et \M3.

\end{itemize}


En examinant les temps d'ex\'ecution sur les trois machines, on peut constater que \TT{Java} est plus rapide que \TT{PpFf}. On verra dans la section~\ref{limitesppff.sect} pourquoi \TT{Java} est plus performant. Par contre, si on compare \TT{PpFf} et \TT{FastFlow}, les diff\'erences des temps d'ex\'ecution sont faibles. Cela d\'emontre que les surco\^uts introduits par \TT{PpFf} par rapport \`a \TT{FastFlow} sont négligeables.

\`A partir des m\^emes s\'eries d'exp\'eriences, nous avons aussi calcul\'e les d\'ebits, soit le nombre de transactions trait\'ees par seconde. Dans la figure~\ref{StockPrice-debits.fig}, on trouve les d\'ebits moyens, indiqués par les valeurs qui composent la courbe sur le graphe, ainsi que des petites barres verticales qui indiquent la dispersion de débits mesurés~; comme pour \TT{WordCount}, cet intervalle indique la moyenne $\pm$ deux fois l'écart-type. Un point int\'eressant, qui peut \^etre observ\'e dans les graphes de d\'ebits, est que la dispersion du d\'ebit pour \TT{Java} est plus grande que celle de \TT{PpFf}. Cela d\'emontre que \TT{PpFf} semble plus stable que \TT{Java}. 

\section{Discussion des résultats et limites de \ppff}
\label{limitesppff.sect}


\begin{figure}
\grapheH{WordCount-temps-java-2001-40-1345}

\grapheH{WordCount-temps-c34581-2003-40-1345}

\caption[Les temps d'exécution des programmes pour \TT{WordCount}
<<optimisé>> sur les machines \M1 et \M3.]{Les temps d'exécution des
programmes pour \TT{WordCount} <<optimisé>> sur les machines \M1 et
\M3. L'axe des $x$ indique le nombre de mots traités. L'axe des $y$
indique le temps d'exécution, en millisecondes.}
\label{WordCount-merged-temps.fig}
\end{figure}

\begin{figure}
\grapheH{WordCount-debits-java-2001-40-1345}

\grapheH{WordCount-debits-c34581-2003-40-1345}

\caption[Les debits des programmes pour \TT{WordCount}
<<optimisé>> sur les machines \M1 et \M3.]{Les debits des
programmes pour \TT{WordCount} <<optimisé>> sur les machines \M1 et
\M3. L'axe des $x$ indique le nombre de mots traités. L'axe des $y$
indique le debits, en millirs de mots traités par second.}
\label{WordCount-merged-debit.fig}
\end{figure}
