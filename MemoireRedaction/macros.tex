% Items, and Enum
\newenvironment{Items}%
{\begingroup\setlength{\topsep}{.050\parsep}\addtolength{\parskip}{-.75\parskip}\setlength{\partopsep}{.25\partopsep}%
\begin{itemize}\setlength{\itemsep}{0pt}\addtolength{\parsep}{-.1\parsep}}%
{\end{itemize}\endgroup\smallbreak}

\newenvironment{Enum}%
{\begingroup\setlength{\topsep}{.050\parsep}\addtolength{\parskip}{-.75\parskip}\setlength{\partopsep}{.25\partopsep}%
\begin{enumerate}\setlength{\itemsep}{0pt}\addtolength{\parsep}{-.1\parsep}}%
{\end{enumerate}\endgroup\smallbreak}

\newcommand{\TT}[1]{{\tt #1}}

\newcommand{\ACOMPLETER}{\footnote{??? \`A compl\'eter!!!}}

\newcommand{\AVERIFIER}[1]{\footnote{??? \`A v\'erifier: #1!!!}}

\newtheorem{definition}{D\'efinition}

% \floatstyle{boxed}
% \newfloat{algorithme}{htbp}{lop}
% \floatname{algorithme}{Algorithme}

% \floatstyle{ruled}
% \newfloat{pseudocode}{htbp}{lop}
% \floatname{pseudocode}{Pseudocode}

% Ne plus utiliser. Utiliser plutot les divers arguments de
% lstlisting, par exemple:
% \begin{lstlisting}[
%   label={[Short caption]Long, long caption.},
%   language=java,
%   gobble=4,
%   caption={...},
%   frame=single,
%   float]

\floatstyle{boxed}
\newfloat{Listing}{htbp}{lop}
\floatname{Listing}{Listing}

\newcommand{\ppff}{\texttt{PpFf}}
\newcommand{\PpFf}{\ppff}

\newcommand{\GT}[1]{{{\textcolor{red}{\footnotesize [[(Guy T.) #1]]}}}}
\newcommand{\gt}[1]{}

\newcommand{\IC}[1]{{\textcolor{blue}{\footnotesize [[(Iulian C.) #1]]}}}
\newcommand{\ic}[1]{}

\lstset{
    language=C++,
    inputencoding=utf8,
    basicstyle=\small\ttfamily,
    frame=none,
    numberstyle=\footnotesize,
    numbersep=5pt,
    tabsize=4,
    literate=
    {?}{{\,c}}1
    {?}{{\^u}}1
    {?}{{\'e}}1
    {?}{{\`e}}1
    {?}{{\^e}}1
    {?}{{\`a}}1
    {?}{{\'E}}1
    {?}{{\`E}}1
    {?}{{\`A}}1
    {`}{{\`{}}}1,
}
