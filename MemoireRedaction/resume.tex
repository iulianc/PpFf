\begin{abstract}

\GT{\`A revoir --- {\bf mais plus tard, apr\`es relecture de
l'ensemble du m\'emoire}.}

\GT{Car pas certain que la premi\`ere phrase soit l'aspect important
sur lequel insister --- plusieurs passages sur les donn\'ees. D'autant
plus que ce n'est pas un aspect sur lequel tu insistes tellement dans
le m\'emoire il me semble. De quelle(s) source(s) t'es-tu
inspir\'ee(s) pour introduire ainsi?}

\GT{Je crois qu'il serait pr\'ef\'erable d'insister sur l'aspect
traitement de flux/s\'equences de donn\'ees avec une API simple,
fonctionnelle, fond\'ee sur une approche <<diviser-pour-r\'egner>>
mais non r\'ecursive.}

\IC{J'ai refait le r\'esum\'e. C'\'etait une copie coll\'ee du r\'esum\'e de ma proposition pour la recherche que je l'ai fait il y a 2 – 3 ann\'ees lorsque j'ai choisi un sujet pour la m\'emoire. Je ne me rappelle pas quelle a \'et\'e ma source d'inspiration.}


Les applications de traitement de flux sont utilis\'ees pour traiter et analyser les donn\'ees qui arrivent de fa\c {c}on continue provenant de sources diff\'erentes. Celles-ci incluent des applications de s\'ecurit\'e, des applications informatiques g\'en\'erant des capteurs, divers types d'applications de surveillance, des applications du domaine de la finance, de la gestion de r\'eseau informatique et des t\'el\'ecommunications. Ces applications sont, dans de nombreux cas, complexes. Leur complexit\'e augmente encore plus lorsque les donn\'ees doivent \^etre trait\'ees en parall\`ele. Afin de traiter de fa\c {c}on simple et efficace les flux de donn\'ees, ce m\'emoire propose une \TT{PpFf}, un \TT{API} simple, fonctionnelle, fond\'ee sur une approche <<diviser-pour-r\'egner>> mais non r\'ecursive. \'Ecrit dans \TT{C++}, l'API permet aux programmeurs d'exposer facilement le parall\'elisme dans des applications de traitement de donn\'ees --- autant du parall\'elisme de flux que du parall\'elisme de donn\'ees --- et ce en obtenant des performances int\'eressantes, ce qui est possible grace \`a une mise en \oe{}uvre qui utilise la biblioth\`eque \TT{FastFlow}, une biblioth\`eque de bas niveau de traitement de flux de donn\'ees en \TT{C++}. 


\end{abstract}

