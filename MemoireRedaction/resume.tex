\begin{abstract}

\GT{\`A revoir --- {\bf mais plus tard, apr\`es relecture de
l'ensemble du m\'emoire}.}

\GT{Car pas certain que la premi\`ere phrase soit l'aspect important
sur lequel insister --- plusieurs passages sur les donn\'ees. D'autant
plus que ce n'est pas un aspect sur lequel tu insistes tellement dans
le m\'emoire il me semble. De quelle(s) source(s) t'es-tu
inspir\'ee(s) pour introduire ainsi?}

\GT{Je crois qu'il serait pr\'ef\'erable d'insister sur l'aspect
traitement de flux/s\'equences de donn\'ees avec une API simple,
fonctionnelle, fond\'ee sur une approche <<diviser-pour-r\'egner>>
mais non r\'ecursive.}




Les logiciels classiques de traitement de donn\'ees sont b\^atis sur le concept de persistance o\`u les donn\'ees, pr\'ealablement stock\'ees, sont interrog\'ees et mises \`a jour plusieurs fois tout au long de leur dur\'ee de vie. Cependant, pour plusieurs applications, les donn\'ees arrivent de fa\c con continue, \`a grande vitesse, et elles doivent \^etre trait\'ees de fa\c con incr\'ementale sans la n\'ecessit\'e de faire plusieurs passages sur les donn\'ees. La complexit\'e de ces types d'applications augmente encore plus lorsque les donn\'ees doivent \^etre trait\'ees en parall\`ele. Afin de traiter de fa\c con simple et efficace les flux de donn\'ees, ce m\`emoire propose une \TT{PpFf}, un API de haut niveau en \TT{C++}. permet aux programmeurs d'exposer facilement le parall\'elisme dans des applications de traitement de donn\'ees --- autant du parall\'elisme de flux que du parall\'elisme de donn\'ees --- et ce en obtenant des performances int\'eressantes, ce qui est possible grace \`a une mise en \oe{}uvre qui utilise la biblioth\`eque \TT{FastFlow}, une biblioth\`eque de bas niveau de traitement de flux de donn\'ees en \TT{C++}.

\end{abstract}

