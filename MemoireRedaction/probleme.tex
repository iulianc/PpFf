 
\chapter{Un chapitre}
\label{un-chap.chap}

\section{Premi\`erement: Tableaux et figures}

fdsfsdfds fdsf

Les travaux de Culler et Sing~\cite{CullerSin99} \ldots (la citation se fait automatiquement avec
{\tt bibtex}). Et voici d'autres
citations~\cite{AndersonEtAl95,Andrews00,AppelbeEtAl96}.


\begin{table}[htbp]
\begin{center}
\begin{tabular}{|l|l|}\hline
x & y \\\hline
y & x \\\hline
\end{tabular}
\end{center}
\caption{L\'egende pour un tableau.}
\label{table.tab}
\end{table}

Un tableau est pr\'esent\'e (Table~\ref{table.tab}).



\begin{figure}[htbp]

\epsffile{Figures/cache.eps}

\caption{L\'egende pour une figure.}
\label{figure-1.fig}
\end{figure}
Comme on le voit \`a la Fig.~\ref{figure-1.fig}, la figure n'est pas
centr\'ee et la taille est d\'etermin\'ee par la taille de la figure
telle que d\'efinie dans le fichier postscript.


\begin{figure}[htbp]

\centerline{\epsfxsize=4cm \epsfbox{Figures/cache.eps}}


\caption{L\'egende pour une autre figure.}
\label{figure-2.fig}
\end{figure}

Par contre, \`a la Fig.~\ref{figure-2.fig}, la figure est centr\'ee et
on peut faire varier sa taille en changeant le param\`etre de {\tt
epsfxsize}.




\section{Deuxi\`emement: macros}


Pour mettre en mode {\tt teletype} = \TT{teletype}.

Pour me faire une note de quelque chose\ACOMPLETER.

Pour me faire une note plus g\'en\'erale\AVERIFIER{V\'erifier cette
partie}

Quelques macros que j'ai d\'efini et que j'utilise r\'eguli\`erement.

S\'erie d'items moins tass\'es:
\begin{itemize}
\item aaa
\item bbb
\item cc
\end{itemize}

S\'erie d'items plus tass\'es --- la diff\'erence est souvent
l\'eg\`ere mais parfois utile lorsqu'il y a une longue s\'erie
d'items:
\begin{Items}
\item aaa
\item bbb
\item cc
\end{Items}

Similaire pour des \'enumerations:
\begin{enumerate}
\item aaa
\item bbb
\item cc
\end{enumerate}

S\'erie d'items plus tass\'es:
\begin{Enum}
\item aaa
\item bbb
\item cc
\end{Enum}


\section{R\'ef\'erences bibliographiques}

Dans un de mes r\'epertoires, qui est public, j'ai toute une s\'erie
de r\'ef\'erences en format \TT{bibtex}, que vous pouvez utiliser. Les
liens sont d\'ej\`a cr\'e\'es dans le fichier \TT{memoire.tex} \`a
l'aide de la commande suivante:
\begin{verbatim}
\bibliography{%
/home/tremblay/biblio/fp,%
/home/tremblay/biblio/mf,%
/home/tremblay/biblio/Tremblay,%
/home/tremblay/biblio/arch+pp,%
biblio-locale
}
\end{verbatim}

Le fichier \TT{biblio-locale} peut \^etre utilis\'e pour ajouter
d'autres r\'ef\'erences qui ne sont pas dans mes bases de donn\'ees.

Toutefois, ces liens ne fonctionneront que si vous utilisez un compte
fonctionnant sour la machine \TT{arabica}.  Si vous n'utilisez pas
\TT{arabica}, vous n'avez qu'\`a mettre ces lignes en commentaire et
\`a ne conserver que la ligne avec \TT{biblio-locale}.

Pour mettre des r\'ef\'erences, on utilise la commande
\TT{cite}. Lorsqu'on met une r\'ef\'erence, ce qui est entre
parenth\`eses ne devrait pas \^etre utilis\'e comme sujet. En d'autres
mots, si on omet ce qui apparait dans les parenth\`eses, la phrase
devrait pouvoir continuer \`a \^etre lisible.

Exemple~: 
\begin{itemize}
\item Les travaux de Lalonde~\cite{Lalonde06url} ont montr\'e que
l'approche par automates cellulaires \'etait int\'eressante.
\end{itemize}

Contre-exemple~: 
\begin{itemize}
\item Les travaux de~\cite{Lalonde06url} ont montr\'e que l'approche
par automates cellulaires \'etait int\'eressante.
\end{itemize}



\section{D\'efinitions}

\begin{definition}
Une d\'efinition importante.
\end{definition}

\begin{definition}[Titre auxiliaire de la d\'efinition]
Une autre d\'efinition importante.
\end{definition}
