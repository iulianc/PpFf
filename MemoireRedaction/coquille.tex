\documentclass[11pt,twoside]{memoireuqam1.3}

\usepackage{graphicx}% Pour les figures
\usepackage[french]{babel}

\usepackage[latin1]{inputenc} % Pour utiliser les caract�res accentu�s sous Unix
%%%%%%%%%%%%%%%%%%%
% ou
% \usepackage[ansinew]{inputenc} % Pour utiliser les caract�res accentu�s sous Windows
% \usepackage[applemac]{inputenc} % Pour utiliser les caract�res accentu�s sous MacOS
%%%%%%%%%%%%%%%%%%%

% Ajouts par Guy T.
% Autres packages souvent utiles
\usepackage{epsfig}
\usepackage{float}
\usepackage{alltt}
\usepackage{graphicx}
\usepackage{url}
%%%%%%%%%%%%%%%%%%%%%%%%%%%%%

\input{macro}  % Pour les d�finitions personnelles

\begin{document}

%%%%%%%%%%%%%%%%%%%%
% Pour la page titre
%%%%%%%%%%%%%%%%%%%%
\title{Mon titre}
\author{Mon nom}
\degreemonth{mois du d�p�t}
\degreeyear{annee du d�p�t}
\uqammemoire   %% Pour la page titre d'un m�moire
% ou
% \uqamthese   %% Pour la page titre d'une th�se
% ou
% \uqamrapport %% Pour la page titre d'un rapport


\thispagestyle{empty}        % La page titre n'est pas num�rot�e
\maketitle

%%%%%%%%%%%%%%%%%%%%
% Page pr�liminaires
%%%%%%%%%%%%%%%%%%%%
%\renewcommand \bibname{R\'EF\'ERENCES}% FACULTATIF
%Enlever le commentaire, si vous voulez qu'apparaisse le titre R�F�RENCES
% plut�t que BIBLIOGRAPHIE

\renewcommand \listfigurename{LISTE DES FIGURES}
\renewcommand \appendixname{APPENDICE}
\renewcommand \figurename{Figure}
\renewcommand \tablename{Tableau}

\pagenumbering{roman} % num�rotation des pages en chiffres romains
\addtocounter{page}{1} % Pour que les remerciements commencent � la page ii

   \parskip=0pt
   \vspace*{0.1 truecm} 
   \begin{center}
    {\uppercase { REMERCIEMENTS }}\par
   \end{center}
   \nobreak \vspace*{1.10 truecm}
   \parskip=2ex



Tout grand accomplissement est rarement r\'ealis\'e par une seule personne. Celle-ci ne fait pas exception. Je suis reconnaissant \`a tous ceux qui m'ont aid\'e et soutenu pendant ces ann\'ees de travail.

Tout d'abord, je voudrais remercier \`a mon superviseur, le professeur Guy Tremblay. Sans son soutien et son aide, cette th\`ese n'aurait pas pu \^etre termin\'ee. Gr\^ace \`a sa riche exp\`erience, il m'a guid\'e dans mes recherches. Il m'a donn\'e de nombreuses id\'ees, suggestions et critiques de valeur.

Je dois \'egalement un grand merci au professeur Marco Aldinucci pour tout son soutien technique pour l'application de \TT{FastFlow}. 

Je souhaite remercier les ing\'enieurs du syst\`eme des laboratoires informatiques et le personnel administratif  qui ont permis \`a ma th\`ese de se d\'erouler sans obstacle technique ou administratif.

Enfin, je tiens \`a remercier \`a ma famille, qui a toujours \'et\'e l\`a, de m'aider moralement \`a surmonter les difficult\'es que j'ai eues en toutes ces ann\'ees de recherches. 



\tableofcontents % Pour g�n�rer la table des mati�res
\listoffigures % Pour g�n�rer la liste des figures
\listoftables % Pour g�n�rer la liste des tableaux
\begin{abstract}

Les logiciels classiques de traitement de donn\'ees sont b\^atis sur le concept de persistance o\`u les donn\'ees, pr\'ealablement stock\'ees, sont interrog\'ees et mise \`a jour plusieurs fois tout au long de leur dur\'ee de vie. Cependant, pour plusieurs applications, les donn\'ees arrivent de fa\c con continue, \`a grande vitesse, et elles doivent \^etre trait\'ees de fa\c con incr\'ementale sans la possibilit\'e de faire plusieurs passages sur les donn\'ees. La complexit\'e de ces types d'applications augmente encore plus lorsque les donn\'ees sont trait\'ees en parall\`ele. Afin de traiter de fa\c con simple et efficace les flux de donn\'ees, ce m\`emoire propose un \TT{API} de haut niveau dans le langage \TT{C++}. En utilisant la biblioth\`eque \TT{FastFlow}, l'interface permet aux utilisateurs d'exposer facilement le parall\'elisme dans des applications s\'equentielles.
	
	
\end{abstract}


% Utilisez l'environnement  abstract pour r�diger votre r�sum�


%%%%%%%%%%%%%%%%%%%%
% Document principal
%%%%%%%%%%%%%%%%%%%%

\parindent=0ex % Facultatif
% pour qu'il n'y ait pas  de renfoncement � chaque alin�a

\begin{introduction}


%\GT{L'introduction n'est pas un chapitre num\'erot\'e.}

\label{introduction.chap}

Blah blah \ldots

Signalons que l'utilisation du package \TT{inputenc} permet d'utiliser
des caractères accentués normaux, plutôt que des caractères accentués
dans l'ancien style (commandes pour les accents).\footnote{Le même
texte écrit avec les accents en commande serait alors le suivant~:
<<Signalons que l'utilisation du package \TT{fontenc} permet
d'utiliser des caract\`eres accentu\'es normaux, plut\^ot que des
caract\`eres accentu\'es dans l'ancien style (commandes pour les
accents).>>}

On remarquera aussi, dans la note en bas de page, l'utilisation des
guillemets français, appelés aussi parfois des <<chevrons>>, qu'on
doit utiliser plutôt que les ``guillemets anglais''.  Ces guillemets
français s'obtiennent à l'aide des caractères
\TT{<}\TT{<}\ldots\TT{>}\TT{>}.

Une autre façon est de définir la macro suivante, puisqu'avec certains
\emph{packages}, les caract\`eres \TT{<}\TT{<}\ldots\TT{>}\TT{>} ne
semblent pas fonctionner~:
{\small
\begin{verbatim}
  \newcommand{\QUOTE}[1]{\og #1 \fg{}}
\end{verbatim}
} 


\newpage

blah blah blah

\end{introduction}



% Utilisez l'environnement  introduction pour r�diger votre introduction

% NOTE IMPORTANTE (GT): Il vaut mieux utiliser un autre nom que
% "chapitre1", un nom plus significatif/approprie pour le contenu du
% chapitre!

\input{chapitre1}

\input{chapitre2}



\begin{conclusion}
\label{conclusion.chap}

Dans ce m\'emoire, nous avons pr\'esent\'e \TT{PpFf}, une API qui peut \^etre utilis\'ee pour cr\'eer des applications de traitement de flux de donn\'ees. Impl\'ement\'ee par l'interm\'ediaire de la biblioth\`eque \TT{FastFlow} et \TT{C++ 17}, l'\TT{API} exploite les fonctionnalit\'es modernes du \TT{C++} et les concepts de programmation g\'en\'erique.

Con\c{c}ue au-dessus de la biblioth\`eque \TT{FastFlow}, \TT{PpFf} fournit une abstraction de programmation souple et expressive qui facilite le d\'eveloppement d'applications parall\`eles, masquant ainsi non seulement la complexit\'e des m\'ecanismes de traitement concurrent, mais aussi le mod\`ele et le traitement de donn\'ees utilis\'es. L'abstraction dans \TT{PpFf} vise \`a simplifier la mani\`ere dont une s\'equence de donn\'ees est visualis\'ee et trait\'ee, permettant ainsi de r\'eutiliser les m\^emes op\'erateurs dans diff\'erents contextes et de r\'eutiliser les m\^emes algorithmes sur diff\'erents mod\`eles de donn\'ees (par exemple \TT{vector}, \TT{list}, \TT{set} etc.). Ces aspects diff\'erencient notre \TT{API} d'autres \TT{API} qui exposent diff\'erents types de donn\'ees \`a utiliser dans la m\^eme interface, ce qui oblige le d\'eveloppeur \`a r\'e\'evaluer ses op\'erations lorsque le contexte change.

Avec une \TT{API} claire et simple, \TT{PpFf} a \'et\'e con\c{c}u dans un style fonctionnel. Il expose donc \`a l'utilisateur un ensemble de m\'ethodes qui peuvent \^etre cha\^in\'ees afin de transformer les donn\'ees dans un style purement fonctionnel \`a l'aide d'une s\'erie de transformations.


Les r\'esultats d'exp\'eriences montrent que \TT{PpFf} peut fournir des applications de traitement de flux de donn\'ees \`a haut d\'ebit. Les bonnes performances obtenues par \TT{PpFf} sont d\^ues en partie au fait qu'il b\'en\'eficie directement de la conception optimis\'ee de \TT{FastFlow} pour le traitement de flux.

En ex\'ecutants diff\'erents tests, \TT{PpFf} a d\'emontr\'e la facilit\'e avec laquelle son \TT{API} permet de r\'esoudre certains probl\`emes classiques de traitement de flux de donn\'ees. Sa conception permet d'utiliser de nombreux concepts de programmation fonctionnels, par exemple, les \TT{lambdas} ou l'\'evaluation paresseuse. Ces aspects font de \TT{PpFf} un outil puissant.

Dans le chapitre sur les exp\'eriences nous avons \'egalement montr\'e que, dans certains cas, notre \TT{API} pouvait obtenir de meilleurs temps d'ex\'ecution qu'un programme \TT{Java} \'equivalent, avec les \TT{Stream}s introduits en Java 8.0, l'un des outils de traitement de donn\'ees le plus performant de nos jours.

Avec \TT{PpFf}, nous croyons que \TT{C++} a une chance de rivaliser avec \TT{Java} qui est consid\'er\'e souvent comme plus convivial que \TT{C++}. 


\section*{\textbf{Perspectives}}

Les travaux pr\'esent\'es dans ce m\'emoire pourraient \^etre \'etendus dans plusieurs directions, dont certaines sont d\'ecrites ci-dessous.

\textbf{Ajouter de nouvelles m\'ethodes parall\`eles} Afin de construire des requ\^etes compl\`etes sur une s\'equence de donn\'ees, nous pourrions \'etendre \TT{PpFf} pour prendre en charge davantage de m\'ethodes parall\`eles de traitement de flux et de donn\'ees, par exemple, \TT{distinct}, \TT{findAny}, \TT{findFirst}, \TT{forEach}, \TT{of}, etc. Ceci donnerait \`a l'utilisateur une plus grande flexibilit\'e pour la manipulation de donn\'ees.

\textbf{Optimiser la génération du graphe FastFlow} L'une des caract\'eristiques de \TT{PpFf} est que chaque op\'eration dans la cha\^ine de traitement s'ex\'ecute sur un \emph{thread} diff\'erent. Comme on l'a vu dans le chapitre sur les expérimentations, cela peut affecter les performances d'un programme si le travail fait par une op\'eration n'est pas assez important pour compenser les surco\^uts introduits par la cr\'eation de \emph{threads}. Afin d'optimiser le graphe généré, il serait intéressant d'impl\'ementer un module qui fusionne des op\'erations lorsque le travail fait par une op\'eration semble trop court, trop simple. Pour ce faire, il faudrait aussi définir un modèle de coûts, pour estimer cette quantité de travail.

\textbf{Impl\'ementer un syst\`eme distribu\'e} La version actuelle de \TT{PpFf} est con\c{c}ue pour fonctionner uniquement en m\'emoire partag\'ee. Afin d'utiliser l'\TT{API} dans un environnement de donn\'ees massives (\emph{Big Data}), il faudrait impl\'ementer un module qui g\`ere la distribution des flux. Le module devrait g\'erer les probl\`emes li\'es aux syst\`emes distribu\'es tels que la coordination des nœuds, la reprise apr\`es une panne, l'allocation des t\^aches, etc. Ce module serait relativement facile \`a impl\'ementer dans \TT{PpFf} gr\^ace \`a \TT{FastFlow}, qui prend d\'ej\`a en charge l'ex\'ecution sur des syst\`emes distribu\'es.


\end{conclusion}





% Utilisez l'environnement  conclusion pour r�diger votre conclusion

%%%%%%%%%%%%%%%%%%%%
% Page liminaires
%%%%%%%%%%%%%%%%%%%%


\chapter{Code source de l'application \TT{WordCount} en \ppff}
\label{sourceCodeWordCountPpFf.ann}

\gt{Habituellement, pas d'article tel <<Le>> dans un titre de chapitre
ou section\ldots\ mais il peut y avoir des exceptions~;)} 


\lstinputlisting[language=c++,caption={Le code source de l'application \TT{WordCount} en \ppff.}]{WordCount.cpp}

\newpage
%\addtocounter{page}{9} % Nombre de pages que contient l'annexe, si ces
					   % annexes ne sont pas g\'en\'er\'ees en latex.

\chapter{Code source de l'application \TT{WordCount} en \TT{Java}}
\label{sourceCodeWordCountJava.ann}

\lstinputlisting[language=java,caption={Le code source de l'application \TT{WordCount} en \TT{Java}.}]{WordCount.java}

\newpage


% ou               %% avec BibTeX
\bibliographystyle{theseuqam2} % non compatible avec le package natbib
\bibliography{biblio-locale}

% ou               %% avec BibTeX
% \bibliographystyle{theseuqam2-natbib} % compatible  natbib
% \bibliography{stat,jeux}

\end{document}
