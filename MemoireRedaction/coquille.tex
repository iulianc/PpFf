\documentclass[11pt,twoside]{memoireuqam1.3}

\usepackage{graphicx}% Pour les figures
\usepackage[french]{babel}

\usepackage[latin1]{inputenc} % Pour utiliser les caract�res accentu�s sous Unix
%%%%%%%%%%%%%%%%%%%
% ou
% \usepackage[ansinew]{inputenc} % Pour utiliser les caract�res accentu�s sous Windows
% \usepackage[applemac]{inputenc} % Pour utiliser les caract�res accentu�s sous MacOS
%%%%%%%%%%%%%%%%%%%

% Ajouts par Guy T.
% Autres packages souvent utiles
\usepackage{epsfig}
\usepackage{float}
\usepackage{alltt}
\usepackage{graphicx}
\usepackage{url}
%%%%%%%%%%%%%%%%%%%%%%%%%%%%%

\input{macro}  % Pour les d�finitions personnelles

\begin{document}

%%%%%%%%%%%%%%%%%%%%
% Pour la page titre
%%%%%%%%%%%%%%%%%%%%
\title{Mon titre}
\author{Mon nom}
\degreemonth{mois du d�p�t}
\degreeyear{annee du d�p�t}
\uqammemoire   %% Pour la page titre d'un m�moire
% ou
% \uqamthese   %% Pour la page titre d'une th�se
% ou
% \uqamrapport %% Pour la page titre d'un rapport


\thispagestyle{empty}        % La page titre n'est pas num�rot�e
\maketitle

%%%%%%%%%%%%%%%%%%%%
% Page pr�liminaires
%%%%%%%%%%%%%%%%%%%%
%\renewcommand \bibname{R\'EF\'ERENCES}% FACULTATIF
%Enlever le commentaire, si vous voulez qu'apparaisse le titre R�F�RENCES
% plut�t que BIBLIOGRAPHIE

\renewcommand \listfigurename{LISTE DES FIGURES}
\renewcommand \appendixname{APPENDICE}
\renewcommand \figurename{Figure}
\renewcommand \tablename{Tableau}

\pagenumbering{roman} % num�rotation des pages en chiffres romains
\addtocounter{page}{1} % Pour que les remerciements commencent � la page ii

   \parskip=0pt
   \vspace*{0.1 truecm} 
   \begin{center}
    {\uppercase { REMERCIEMENTS }}\par
   \end{center}
   \nobreak \vspace*{1.10 truecm}
   \parskip=2ex


\GT{Je crois qu'un des évaluateurs n'avait pas aimé <<grand
accomplissement>>.  Il semblait trouver que ça faisait trop
<<pompeux>>~:( À modifier!}

\IC{J'ai remplacé le mot accomplissement par projet.}

\GT{Il est aussi d'usage de remercier le directeur de recherche ou
l'organisme subventionnaire lorsque du support financier a été
offert. J'ai ajouté une phrase à ce  sujet.}

\IC{Oui c'est vrai. Je ne crois pas avoir révié cette section.}

Tout grand projet est rarement r\'ealis\'e par une seule personne. Celui-ci ne fait pas exception. Je suis reconnaissant \`a tous ceux et celles qui m'ont aid\'e et soutenu pendant ces ann\'ees de travail.

Tout d'abord, je voudrais remercier mon directeur de recherche, le professeur Guy Tremblay. Sans son soutien et son aide, ce m\'emoire n'aurait pas pu \^etre termin\'e. Gr\^ace \`a sa riche exp\`erience, il m'a guid\'e dans mes recherches. Il m'a donn\'e de nombreuses id\'ees, suggestions et critiques constructives.
%
Je le remercie aussi pour son support financier au cours de l'hiver et de l'été 2020.

Je dois \'egalement un grand merci au professeur Marco Aldinucci, de l'Universit\'e de Turin, pour son soutien technique pour l'application de \TT{FastFlow}. 

Je souhaite aussi remercier les analystes et techniciens syst\`emes des laboratoires informatiques et le personnel administratif qui ont permis \`a mon m\'emoire de se d\'erouler sans obstacle technique ou administratif.

Enfin, je tiens \`a remercier \`a ma famille, qui a toujours \'et\'e l\`a, m'ayant aid\'e moralement \`a surmonter les difficult\'es que j'ai eues durant ces ann\'ees de recherche. 



\tableofcontents % Pour g�n�rer la table des mati�res
\listoffigures % Pour g�n�rer la liste des figures
\listoftables % Pour g�n�rer la liste des tableaux
\begin{abstract}

Les logiciels classiques de traitement de donn\'ees sont b\^atis sur le concept de persistance o\`u les donn\'ees, pr\'ealablement stock\'ees, sont interrog\'ees et mise \`a jour plusieurs fois tout au long de leur dur\'ee de vie. Cependant, pour plusieurs applications, les donn\'ees arrivent de fa\c con continue, \`a grande vitesse, et elles doivent \^etre trait\'ees de fa\c con incr\'ementale sans la possibilit\'e de faire plusieurs passages sur les donn\'ees. La complexit\'e de ces types d'applications augmente encore plus lorsque les donn\'ees sont trait\'ees en parall\`ele. Afin de traiter de fa\c con simple et efficace les flux de donn\'ees, ce m\`emoire propose un \TT{API} de haut niveau dans le langage \TT{C++}. En utilisant la biblioth\`eque \TT{FastFlow}, l'interface permet aux utilisateurs d'exposer facilement le parall\'elisme dans des applications s\'equentielles.
	
	
\end{abstract}


% Utilisez l'environnement  abstract pour r�diger votre r�sum�


%%%%%%%%%%%%%%%%%%%%
% Document principal
%%%%%%%%%%%%%%%%%%%%

\parindent=0ex % Facultatif
% pour qu'il n'y ait pas  de renfoncement � chaque alin�a

\begin{introduction}

Le traitement de flux de donn\'ees devient de plus en plus important en raison de la grande quantit\'e de donn\'ees continuellement g\'en\'er\'ees, provenant de diverses sources telles que des capteurs, des indicateurs boursiers, des dispositifs de r\'eseau, etc. Afin de traiter rapidement une grande quantit\'e de donn\'ees, notamment en exploitant les capacit\'es des processeurs multicœurs modernes, une application doit \^etre con\c{c}ue en parall\`ele. La conception et la mise en œuvre d'applications parall\`eles efficientes pour traiter des flux de donn\'ees posent des d\'efis aux d\'eveloppeurs. Les co\^uts de communications~\citep{amarasinghe2011ascr}, les conditions de course (\emph{data race})~\citep{wu2015detecting}, les interblocages~\citep{haque2006concurrent} et les d\'es\'equilibres de charge de travail entre les fils d'ex\'ecution~\citep{amarasinghe2011ascr} sont quelques exemples de probl\`emes qui demandent des efforts suppl\'ementaires aux programmeurs. De plus, la complexit\'e du code peut diminuer la productivit\'e et, par cons\'equent, augmenter les co\^uts de d\'eveloppement. Ce m\'emoire vise \`a proposer une bibliothèque, en C++, qui permet traiter de fa\c{c}on simple les flux de donn\'ees en tentant de dissimuler cette complexit\'e.

\section*{D\'efinition du probl\`eme}

Une application qui traite un flux de données peut \^etre consid\'er\'ee comme un pipeline \`a travers lequel les donn\'ees du flux sont produites, trait\'ees et consomm\'ees en continu. Le parall\'elisme dans le traitement d'un tel flux consiste \`a traiter des \'el\'ements distinct du flux de fa\c{c}on concurrente --- \emph{parall\'elisme de flux} (\emph{flow parallelism} ou \emph{stream parallelism}) --- et \`a r\'epliquer un m\^eme op\'erateur sur des sous-groupes d'\'el\'ements du flux --- \emph{parall\'elisme de donn\'ees} (\emph{data parallelism}). La transformation, le tri ou la r\'eduction par un op\'erateur binaire sont des exemples d'op\'erations pour lesquelles du parall\'elisme de donn\'ees peut \^etre exploit\'e. 


Les applications parall\`eles sont g\'en\'eralement cod\'ees \`a l'aide de biblioth\`eques comme \TT{FastFlow}~\citep{AldinucciEtAl14} ou \TT{TBB}~\citep{Reinders07}. Ces outils permettent aux utilisateurs d'implémenter des solutions robustes et portables avec une abstraction de haut niveau qui masque la complexit\'e des m\'ecanismes de concurrence, tels que la gestion des \emph{threads} et les synchronisations. Bien que les mod\`eles offerts par ces outils aient pour but de simplifier le d\'eveloppement d'applications parall\`eles, ils n'offrent pas une interface standard. Notre bibliothèque veut introduire un mod\`ele simple, o\`u le programmeur cr\'ee un pipeline par l'interm\'ediaire d'une interface (API, \emph{Application Programming Interface}) qui expose clairement les op\'erations de transformation, et o\`u chaque transformation est relativement simple --- ce qui correspond \`a une application du
principe <<diviser-pour-r\'egner>> mais de fa\c{c}on \emph{non r\'ecursive}.

R\'e\'ecrire une application est g\'en\'eralement co\^uteux en termes de temps de d\'eveloppement. Les approches visant \`a pallier ce d\'efaut sont souvent des outils de programmations parall\`eles d\'evelopp\'es pour introduire le parall\'elisme dans du code s\'equentiel existant. Des exemples sont \TT{OpenMP}~\citep{ChandraEtAl01} et \TT{OpenACC}~\citep{farber2016parallel} qui utilisent une approche bas\'ee sur l'ajout de directives --- des commentaires sp\'eciaux trait\'es par le compilateur ---~ et \TT{Cilk}~\citep{leiserson1998programming} qui est une extension simple du langage~\TT{C}. Malheureusement, ces outils ne sont pas bien adapt\'es au traitement de flux de donn\'ees.

Ces derni\`eres ann\'ees, plusieurs biblioth\`eques \'ecrites en \emph{C++} ont \'et\'e con\c{c}ues pour traiter des flux de donn\'ees. Parmi les plus r\'ecentes, on retrouve \TT{RaftLib}~\citep{beard2017raftlib}, \TT{StarPU}~\citep{starPuReferenceEnLigne} et~\TT{SkePU}~\citep{skePuReferenceEnLigne}. Malgr\'e le fait que ces bibliothèques soient des outils performants, aucune d'entre elle ne facilite la description des op\'erations d'un flux comme le permet le cha\^inage des op\'erations.

%, comme c'est possible avec Spark~\citep{apachSpark} ou les \emph{Stream} de Java~8~\citep{javaStreamAPI}.

Des outils qui supportent le traitement de flux de donn\'ees sont disponibles aussi dans d'autres langages de programmation que \TT{C}/\TT{C++}. Parmi les plus connus on retrouve \TT{Spark}~\citep{frampton2015mastering} (Java, Scala et autres langages), les \TT{Stream}s de \TT{Java 8}~\citep{warburton2014java} et \TT{Flink}~\citep{flinkReferenceEnLigne} (Java et Scala). Notre bibliothèque possède une \TT{API} semblable \`a celle des \TT{Stream}s de \TT{Java}~8, mais en \TT{C++}, et elle est souvent plus simple \`a sp\'ecifier notamment gr\^ace aux \emph{templates}.



\section*{Objectifs}


Dans le contexte du langage de programmation \TT{C++}, il existe plusieurs biblioth\`eques qui offrent des algorithmes de traitement de flux de donn\'ees. Cependant, plusieurs des constructions pour exprimer des algorithmes parall\`eles se limitent \`a des op\'erations de transformation et et de r\'eduction. Ce m\'emoire a comme objectif d'enrichir ces op\'erations avec de nouvelles op\'erations, et ce dans une interface simple \`a utiliser. Ceci, associ\'e \`a de nouvelles fonctionnalit\'es telles que les expressions lambda~\citep{josuttis2012c++}, aide un programmeur \`a \'ecrire des op\'erations complexes pour un flux de donn\'ees. Cette nouvelle bibliothèque en \TT{C++}, appel\'ee \TT{PpFF},  est mise en \oe{}uvre avec la biblioth\`eque \TT{FastFlow}.


\'Etant donn\'e la ressemblance avec les \TT{Stream}s de \TT{Java~8},
les performances de \TT{PpFf} seront \'evalu\'ees en les comparant \`a celles des \TT{Stream}s. Comme nous le verrons, les r\'esultats indiquent que \TT{PpFf} peut en effet traiter des donn\'ees \`a haut d\'ebit.
%
Nous comparerons aussi les performances de \ppff\ avec celles de
\TT{FastFlow}, afin d'analyser les surco\^uts introduits par rapport à \TT{FastFlow}.
%
En plus de mesurer les performances de notre bibliothèque, nous illustrerons \'egalement son expressivit\'e en impl\'ementant certains cas d'utilisation typiques rencontr\'es dans les applications de traitement de flux de donn\'ees.


\section*{Organisation du m\'emoire}

Les chapitres qui forment le c\oe{}ur du m\'emoire sont organis\'es
comme suit.


Le chapitre~\ref{outils_connus.chap} \nameref{outils_connus.chap} pr\'esente des outils existants portant sur le traitement des flux de donn\'ees.  Tout d'abord, il introduit les architectures utilis\'ees par divers outils, puis il pr\'esente les mod\`eles de programmations permettant d'exprimer les traitements de donn\'ees.

Le chapitre~\ref{description.chap} \nameref{description.chap} pr\'esente les m\'ethodes de notre bibliothèque. Un r\'esum\'e sous forme de tableau de toutes les m\'ethodes implément\'ees est pr\'esent\'e en annexe. À l'aide de quelques exemples, le chapitre d\'ecrit aussi plus en d\'etail les m\'ethodes les plus importantes de la bibliothèque.

Le chapitre~\ref{implementation.chap} \nameref{implementation.chap} explique comment nous avons mis en œuvre notre bibliothèque en utilisant la biblioth\`eque de bas niveau \TT{FastFlow}.

Finalement, le chapitre~\ref{experiences.chap} \nameref{experiences.chap} pr\'esente une \'evaluation des performances sur deux cas d'utilisation typiques.

\end{introduction}



% Utilisez l'environnement  introduction pour r�diger votre introduction

% NOTE IMPORTANTE (GT): Il vaut mieux utiliser un autre nom que
% "chapitre1", un nom plus significatif/approprie pour le contenu du
% chapitre!

\input{chapitre1}

\input{chapitre2}



\begin{conclusion}
\label{conclusion.chap}

Dans ce m\'emoire, nous avons pr\'esent\'e \TT{PpFf}, une API qui peut \^etre utilis\'ee pour cr\'eer des applications de traitement de flux de donn\'ees. Impl\'ement\'ee par l'interm\'ediaire de la biblioth\`eque \TT{FastFlow} et \TT{C++ 17}, l'\TT{API} exploite les fonctionnalit\'es modernes du \TT{C++} et les concepts de programmation g\'en\'erique.

Con\c cue au-dessus de la biblioth\`eque \TT{FastFlow}, \TT{PpFf} fournit une abstraction de programmation souple et expressive qui facilite le d\'eveloppement d'applications parall\`eles, masquant ainsi non seulement la complexit\'e des m\'ecanismes de traitement concurrent, mais aussi le mod\`ele et le traitement de donn\'ees utilis\'es. L'abstraction dans \TT{PpFf} vise \`a simplifier la mani\`ere dont une s\'equence de donn\'ees est visualis\'ee et trait\'ee, permettant ainsi de r\'eutiliser les m\^emes op\'erateurs dans diff\'erents contextes et de r\'eutiliser les m\^emes algorithmes sur diff\'erents mod\`eles de donn\'ees (par exemple \TT{vector}, \TT{list}, \TT{set} etc.). Ces aspects diff\'erencient notre \TT{API} d'autres \TT{API} qui exposent diff\'erents types de donn\'ees \`a utiliser dans la m\^eme interface, ce qui oblige le d\'eveloppeur \`a r\'e\'evaluer ses op\'erations lorsque le contexte change.

Avec une \TT{API} claire et simple, \TT{PpFf} a \'et\'e con\c cu dans un style fonctionnel. Il expose donc \`a l'utilisateur un ensemble de m\'ethodes qui peuvent \^etre cha\^in\'ees afin de transformer les donn\'ees dans un style purement fonctionnel \`a l'aide d'une s\'erie de transformations.


Les r\'esultats d'exp\'eriences montrent que \TT{PpFf} peut fournir des applications de traitement de flux de donn\'ees \`a haut d\'ebit. Les bonnes performances obtenues par \TT{PpFf} sont d\^ues en partie au fait qu'il b\'en\'eficie directement de la conception optimis\'ee de \TT{FastFlow} pour le traitement de flux.

En ex\'ecutants diff\'erents tests, \TT{PpFf} a d\'emontr\'e la facilit\'e avec laquelle son \TT{API} permet de r\'esoudre certains probl\`emes classiques de traitement de flux de donn\'ees. Sa conception permet d'utiliser de nombreux concepts de programmation fonctionnels, par exemple, les \TT{lambdas} ou l'\'evaluation paresseuse. Ces aspects font de \TT{PpFf} un outil puissant.

Dans le chapitre sur les exp\'eriences nous avons \'egalement montr\'e que, dans certains cas, notre \TT{API} pouvait obtenir de meilleurs temps d'ex\'ecution qu'un programme \TT{Java} \'equivalent, avec les \TT{Stream}s introduits en Java 8.0, l'un des outils de traitement de donn\'ees le plus performant de nos jours.

Avec \TT{PpFf}, nous croyons que \TT{C++} a une chance de rivaliser avec \TT{Java} qui est consid\'er\'e souvent comme plus convivial que \TT{C++}. 


\section*{\textbf{Perspectives}}

Les travaux pr\'esent\'es dans ce m\'emoire pourraient \^etre \'etendus dans plusieurs directions, dont certaines sont d\'ecrites ci-dessous.

\textbf{Ajouter de nouvelles m\'ethodes parall\`eles} Afin de construire des requ\^etes compl\`etes sur une s\'equence de donn\'ees, nous pourrions \'etendre \TT{PpFf} pour prendre en charge davantage de m\'ethodes parall\`eles de traitement de flux et de donn\'ees, par exemple, \TT{distinct}, \TT{findAny}, \TT{findFirst}, \TT{forEach}, \TT{of}, etc. Ceci donnerait \`a l'utilisateur une plus grande flexibilit\'e pour la manipulation de donn\'ees.

\textbf{Impl\'ementer un syst\`eme distribu\'e} La version actuelle de \TT{PpFf} est con\c cue pour fonctionner uniquement en m\'emoire partag\'ee. Afin d'utiliser l'\TT{API} dans un environnement de donn\'ees massives (\emph{Big Data}), il faudrait impl\'ementer un module qui g\`ere la distribution des flux. Le module devrait g\'erer les probl\`emes li\'es aux syst\`emes distribu\'es tels que la coordination des nœuds, la reprise apr\`es une panne, l'allocation des t\^aches, etc. Ce module serait relativement facile \`a impl\'ementer dans \TT{PpFf} gr\^ace \`a \TT{FastFlow}, qui prend d\'ej\`a en charge l'ex\'ecution sur des syst\`emes distribu\'es.


\end{conclusion}





% Utilisez l'environnement  conclusion pour r�diger votre conclusion

%%%%%%%%%%%%%%%%%%%%
% Page liminaires
%%%%%%%%%%%%%%%%%%%%


\chapter{Code source de l'application \TT{WordCount} en \ppff}
\label{sourceCodeWordCountPpFf.ann}

\gt{Habituellement, pas d'article tel <<Le>> dans un titre de chapitre
ou section\ldots\ mais il peut y avoir des exceptions~;)} 


\lstinputlisting[language=c++,caption={Le code source de l'application \TT{WordCount} en \ppff.}]{WordCount.cpp}

\newpage
%\addtocounter{page}{9} % Nombre de pages que contient l'annexe, si ces
					   % annexes ne sont pas g\'en\'er\'ees en latex.

\chapter{Le code source de l'application WordCount dans Java}
\label{sourceCodeWordCountJava.ann}

\lstinputlisting[language=java,caption={Le code source de l'application WordCount}]{WordCount.java}

\newpage


% ou               %% avec BibTeX
\bibliographystyle{theseuqam2} % non compatible avec le package natbib
\bibliography{biblio-locale}

% ou               %% avec BibTeX
% \bibliographystyle{theseuqam2-natbib} % compatible  natbib
% \bibliography{stat,jeux}

\end{document}
