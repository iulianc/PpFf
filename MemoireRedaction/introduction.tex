
\chapter{Introduction}

\label{introduction.chap}



Blah blah \ldots


Signalons que l'utilisation du package \TT{inputenc} permet d'utiliser
des caractères accentués normaux, plutôt que des caractères accentués
dans l'ancien style (commandes pour les accents).\footnote{Le même
texte écrit avec les accents en commande serait alors le suivant~:
<<Signalons que l'utilisation du package \TT{fontenc} permet
d'utiliser des caract\`eres accentu\'es normaux, plut\^ot que des
caract\`eres accentu\'es dans l'ancien style (commandes pour les
accents).>>}

On remarquera aussi, dans la note en bas de page, l'utilisation des
guillemets français, appelés aussi parfois des <<chevrons>>, qu'on
doit utiliser plutôt que les ``guillemets anglais''.  Ces guillemets
français s'obtiennent à l'aide des caractères
\TT{<}\TT{<}\ldots\TT{>}\TT{>}.

Une autre façon est de définir la macro suivante, puisqu'avec certains
\emph{packages}, les caract\`eres \TT{<}\TT{<}\ldots\TT{>}\TT{>} ne
semblent pas fonctionner~:
{\small
\begin{verbatim}
  \newcommand{\QUOTE}[1]{\og #1 \fg{}}
\end{verbatim}
} 

