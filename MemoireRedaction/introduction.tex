\begin{introduction}

Le traitement de flux de donn\'ees devient de plus en plus important en raison de la grande quantit\'e de donn\'ees continuellement g\'en\'er\'ees, provenant de diverses sources telles que des capteurs, des indicateurs boursiers, des dispositifs de r\'eseau, etc. Afin de traiter rapidement une grande quantit\'e de donn\'ees, notamment en exploitant les capacit\'es des processeurs multicœurs modernes, une application doit \^etre con\c cue en parall\`ele. La conception et la mise en œuvre d'applications parall\`eles efficientes pour traiter des flux de donn\'ees posent des d\'efis aux d\'eveloppeurs. Les co\^uts de communications~\citep{amarasinghe2011ascr}, les conditions de course (\emph{data race})~\citep{wu2015detecting}, les interblocages~\citep{haque2006concurrent} et les d\'es\'equilibres de charge de travail entre les fils d'ex\'ecution~\citep{amarasinghe2011ascr} sont quelques exemples de probl\`emes qui demandent des efforts suppl\'ementaires aux programmeurs. De plus, la complexit\'e du code peut diminuer la productivit\'e et, par cons\'equent, augmenter les co\^uts de d\'eveloppement. Ce m\'emoire vise \`a proposer une API (\emph{Application Programming Interface}), en C++, qui permet traiter de fa\c con simple et efficace les flux de donn\'ees en tentant de dissimuler cette complexit\'e.

\section*{D\'efinition du probl\`eme}

Une application qui traite un flux de données peut \^etre consid\'er\'ee comme un pipeline \`a travers lequel les donn\'ees du flux sont produites, trait\'ees et consomm\'ees en continu. Le parall\'elisme dans le traitement d'un tel flux consiste \`a traiter des \'el\'ements distinct du flux de fa\c{c}on concurrente --- \emph{parall\'elisme de flux} (\emph{flow parallelism} ou \emph{stream parallelism}) --- et \`a r\'epliquer un m\^eme op\'erateur sur des sous-groupes d'\'el\'ements du flux --- \emph{parall\'elisme de donn\'ees} (\emph{data parallelism}). La transformation, le tri ou la r\'eduction par un op\'erateur binaire sont des exemples d'op\'erations pour lesquelles du parall\'elisme de donn\'ees peut \^etre exploit\'e. 


Les applications parall\`eles sont g\'en\'eralement cod\'ees \`a l'aide de biblioth\`eques comme \TT{FastFlow}~\citep{AldinucciEtAl14} ou \TT{TBB}~\citep{Reinders07}. Ces outils permettent aux utilisateurs d'implémenter des solutions robustes et portables avec une abstraction de haut niveau qui masque la complexit\'e des m\'ecanismes de concurrence, tels que la gestion des \emph{threads} et les synchronisations. Bien que les mod\`eles offerts par ces outils aient pour but de simplifier le d\'eveloppement d'applications parall\`eles, ils n'offrent pas une interface standard. Notre API veut introduire un mod\`ele simple, o\`u le programmeur cr\'ee un pipeline par l'interm\'ediaire d'une interface qui expose clairement les op\'erations de transformation, et o\`u chaque transformation est relativement simple --- ce qui correspond \`a une application du
principe <<diviser-pour-r\'egner>> mais de fa\c{c}on non r\'ecursive.

R\'e\'ecrire une application est g\'en\'eralement co\^uteux en termes de temps de d\'eveloppement. Les approches visant \`a pallier ce d\'efaut sont souvent des outils de programmations parall\`eles d\'evelopp\'es pour introduire le parall\'elisme dans du code s\'equentiel existant. Des exemples sont \TT{OpenMP}~\citep{ChandraEtAl01} et \TT{OpenACC}~\citep{farber2016parallel} qui utilisent une approche bas\'ee sur l'ajout de directives --- des commentaires sp\'eciaux trait\'es par le compilateur ---~ et \TT{Cilk}~\citep{leiserson1998programming} qui est une extension simple du langage~\TT{C}. Malheureusement, ces outils ne sont pas bien adapt\'es au traitement de flux de donn\'ees.

Ces derni\`eres ann\'ees, plusieurs biblioth\`eques \'ecrites en \emph{C++} ont \'et\'e con\c{c}ues pour traiter des flux de donn\'ees. Parmi les plus r\'ecentes, on retrouve \TT{RaftLib}~\citep{beard2017raftlib}, \TT{StarPU}~\citep{starPuReferenceEnLigne} et~\TT{SkePU}~\citep{skePuReferenceEnLigne}. Malgr\'e le fait que ces bibliothèques soient des outils performants, aucune d'entre elle ne facilite la description des op\'erations d'un flux comme le permet le cha\^inage des op\'erations.

%, comme c'est possible avec Spark~\citep{apachSpark} ou les \emph{Stream} de Java~8~\citep{javaStreamAPI}.

Des outils qui supportent le traitement de flux de donn\'ees sont disponibles aussi dans d'autres langages de programmation que \TT{C}/\TT{C++}. Parmi les plus connus on retrouve \TT{Spark}~\citep{frampton2015mastering} (Java, Scala et autres langages), les \TT{Stream}s de \TT{Java 8}~\citep{warburton2014java} et \TT{Flink}~\citep{flinkReferenceEnLigne} (Java et Scala). Notre \TT{API} est semblable \`a celle des \TT{Stream}s de \TT{Java}~8, mais en \TT{C++}, et elle est souvent plus simple \`a sp\'ecifier notamment gr\^ace aux \emph{templates}.



\section*{Objectifs}


Dans le contexte du langage de programmation \TT{C++}, il existe plusieurs biblioth\`eques qui offrent des algorithmes de traitement de flux de donn\'ees. Cependant, plusieurs des constructions pour exprimer des algorithmes parall\`eles se limitent \`a des op\'erations de transformation et et de r\'eduction. Ce m\'emoire a comme objectif d'enrichir ces op\'erations avec de nouvelles op\'erations, et ce dans une interface simple \`a utiliser. Ceci, associ\'e \`a de nouvelles fonctionnalit\'es telles que les expressions lambda~\citep{josuttis2012c++}, aide un programmeur \`a \'ecrire des op\'erations complexes pour un flux de donn\'ees. Cette nouvelle \TT{API} en \TT{C++}, appel\'ee \TT{PpFF},  est mise en \oe{}uvre avec la biblioth\`eque \TT{FastFlow}.


\'Etant donn\'e la ressemblance avec les \TT{Stream}s de \TT{Java~8},
les performances de \TT{PpFf} seront \'evalu\'ees en les comparant \`a celles des \TT{Stream}s. Comme nous le verrons, les r\'esultats indiquent que \TT{PpFf} peut en effet traiter des donn\'ees \`a haut d\'ebit.
%
Nous comparerons aussi les performances de \ppff\ avec celles de
\TT{FastFlow}, afin d'analyser les surco\^uts introduits par notre API.
%
En plus de mesurer les performances de notre API, nous illustrerons \'egalement son expressivit\'e en impl\'ementant certains cas d'utilisation typiques rencontr\'es dans les applications de traitement de flux de donn\'ees.


\section*{Organisation du m\'emoire}

Les chapitres qui forment le c\oe{}ur du m\'emoire sont organis\'es
comme suit.


Le chapitre~\ref{outils_connus.chap} \nameref{outils_connus.chap} pr\'esente les outils existants portant sur le traitement des flux de donn\'ees.  Tout d'abord, il introduit les architectures utilis\'ees par divers outils, puis il pr\'esente les mod\`eles de programmations permettant d'exprimer les traitements de donn\'ees.

Le chapitre~\ref{description.chap} \nameref{description.chap} pr\'esente les m\'ethodes de notre \TT{API}. Un r\'esum\'e sous forme de tableau de toutes les m\'ethodes implément\'ees est pr\'esent\'e au d\'ebut du chapitre. Accompagn\'e de quelques exemples, le chapitre d\'ecrit aussi plus en d\'etail les m\'ethodes les plus importantes.

Le chapitre~\ref{implementation.chap} \nameref{implementation.chap} examine comment nous avons mis en œuvre notre \TT{API} en utilisant la biblioth\`eque de bas niveau \TT{FastFlow}.

Finalement, le chapitre~\ref{experiences.chap} \nameref{experiences.chap} pr\'esente une \'evaluation des performances sur deux cas d'utilisation typiques.

\end{introduction}

