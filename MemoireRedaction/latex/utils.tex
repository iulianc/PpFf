% fonts and lang packages
\usepackage[utf8]{inputenc} 	% utf8
\usepackage[T1]{fontenc} 	% french characters
\usepackage[french]{babel}	% french dictionnary

% utils
\usepackage{graphicx} 	% enhenced includes
\usepackage{latex/comment}	% comment blocks
\usepackage[hyphens]{url}		% display url links
\usepackage{rotating}	% rotating figues
%\usepackage{authblk}	% enhenced authors
\usepackage{color}		% colors
\usepackage{float}

\usepackage{tablefootnote}
\usepackage{ifthen}
\usepackage{tikz, tikz-qtree}

% tables
\usepackage{tabularx}
\usepackage{latex/multirow}

\usepackage{lmodern}
\usepackage{microtype}
\usepackage{natbib}

\usepackage{xcolor}
\usepackage[export]{adjustbox}
\definecolor{light-gray}{gray}{0.8}

% \usepackage{relsize}
\usepackage{subcaption}
\captionsetup[figure]{labelfont=bf}
\captionsetup[table]{labelfont=bf}

% renew authors
%\renewcommand\Authand{ et }
%\renewcommand\Authands{, et }

% listings
\usepackage{listings}

\lstset{
	inputencoding=utf8,
	basicstyle=\small\ttfamily,
	frame=none,
	numberstyle=\footnotesize,
	numbersep=5pt,
	tabsize=4,
	literate=
	{ç}{{\,c}}1
	{û}{{\^u}}1
	{é}{{\'e}}1
	{è}{{\`e}}1
	{ê}{{\^e}}1
	{à}{{\`a}}1
	{É}{{\'E}}1
	{È}{{\`E}}1
	{À}{{\`A}}1
	{`}{{\`{}}}1,
}

\usepackage{pdfpages}

\usepackage{nameref}
\usepackage{lscape}

% Change paragraph style
\usepackage{latex/titlesec}
\titleformat{\paragraph}[runin]
{\bfseries}{\theparagraph}{1em}{}

% maths packages
\usepackage{amsmath}
\usepackage{amssymb}
\usepackage{amsthm}

% Change asm questions styles
\makeatletter
\newtheoremstyle{indented}
  {3pt}% space before
  {3pt}% space after
  {\addtolength{\@totalleftmargin}{2.5em}
   \addtolength{\linewidth}{-2.5em}
   \parshape 1 2.5em \linewidth}% body font
  {}% indent
  {\bfseries}% header font
  { - }% punctuation
  {0em}% after theorem header
  {}% header specification (empty for default)
\def\th@plain{%
  \thm@notefont{}% same as heading font
  \itshape % body font
}
\def\th@definition{%
  \thm@notefont{}% same as heading font
  \normalfont % body font
}
\makeatother

\theoremstyle{indented}
\newtheorem{question}{Question}[section]
\newtheorem*{response}{Réponse}

\usepackage{latex/silence}
\WarningFilter{frenchb.ldf}{Figure}

% J'ai de la difficult\'e \`a lire quand les hyperliens sont vert
% pale. De plus, je crois que quand tu imprimeras les versions
% papier... cela risque de g\'en\'erer du texte plus pale, moins
% lisible... Donc j'ai desactive la couleur pour les liens... mais
% c'est facile de les reactiver: voir makefile.

\newboolean{ColoredLinks}
\setboolean{ColoredLinks}{true}


\usepackage[pdfborder={0 0 0}, colorlinks=true]{hyperref}
% compatibility between pdftex and hyperref
\ifthenelse{\boolean{ColoredLinks}}
{ \hypersetup{citecolor=blue} }
{ \hypersetup{colorlinks=false} }

%%%%%%%%%%%%%%%%%%%%%%%%%%%%%%%%%%%%%%%%%%%%%
%%%%%%%%%%%%%%% Macros et ajouts %%%%%%%%%%%%
%%%%%%%%%%%%%%%%%%%%%%%%%%%%%%%%%%%%%%%%%%%%%
\floatstyle{ruled}

\newfloat{programme}{htbp}{lop}[chapter]
\floatname{programme}{Programme}

\newcommand{\ACOMPLETER}[1]{{\textcolor{red}{#1}}}
\newcommand{\AVERIFIER}[1]{{\textcolor{blue}{#1}}}

\newcommand{\todo}[1]{{\textcolor{blue}{<\footnotesize A faire: #1>}}}
\newcommand{\TODO}[1]{{\textcolor{blue}{[[A FAIRE: #1]]}}}

\newcommand{\README}{{\sc readme}}
\newcommand{\DOCDOWN}{{\texttt{DocDown}}}
\newcommand{\MAN}{{\sc man}}

\newcommand\rot{\rotatebox{90}}

\newcommand{\Legende}[1]{{\footnotesize{\newline L\'egende: #1}}}

\newcommand{\B}{{$\bullet$}}

\newcommand{\subsectionSansNum}[1]{\paragraph{#1}{\ }}

% Pour presentation de sous-sections avec numero alphabetique
\newcounter{numSoussectionAlpha}
\setcounter{numSoussectionAlpha}{1}

\newcommand{\resetsoussectionAlpha}{\setcounter{numSoussectionAlpha}{1}}
\renewcommand{\thenumSoussectionAlpha}{\Alph{numSoussectionAlpha}}
\newcommand{\soussectionAlpha}[1]{\thenumSoussectionAlpha.~{\bf #1}{\ }\stepcounter{numSoussectionAlpha}}
\renewcommand{\soussectionAlpha}[1]{{\bf #1}{\ }\stepcounter{numSoussectionAlpha}}
%\renewcommand{\soussectionAlpha}[1]{{\bf #1}{\ }\stepcounter{numSoussectionAlpha}}

\newcommand{\captionEcosysteme}[1]{\caption[Place #1 dans l'\'ecosyst\`eme de documentation Nit.]{Place #1 dans l'\'ecosyst\`eme de documentation Nit.}}

\newcommand{\captionTaxonomieThemes}[1]{\caption[Taxonomie des
 th\`emes #1 les plus fréquents dans les fichiers \README.]{Taxonomie des
 th\`emes #1 les plus fréquents dans les fichiers \README.
 \Legende{\textbf{occ.}: nombre d'occurrences du thème dans le corpus;
 \textbf{\% occ.}: pourcentage d'occurrences associées à ce thème;
 \textbf{moy.}: moyenne du nombre d'occurrences du thème par document;
 \textbf{\# doc.}: nombre de documents contenant le thème;
 \textbf{\% doc.}: pourcentage de documents contenant le thème;
 \textbf{\% pos.}: position moyenne du thème dans les documents en pourcentage de la taille du document.
 Les thèmes sont triés par position moyenne dans les documents.}}}

\newcommand{\captionComparaisonExpert}[3]{\caption[Comparaison du
nombre #1 dans le fichier \README\ produit par l'expert~#2 pour la biblioth\`eque \texttt{#3}.]{Comparaison du
nombre #1 dans le fichier \README\ produit par l'expert~#2 pour la biblioth\`eque \texttt{#3} avec ceux issus de
GitHub analys\'es dans le chapitre~\ref{readmes-github} et ceux
\'ecrits par les experts Nit pour le corpus d'alignement du
chapitre~\ref{align}.}}

\newtheorem{definition}{D\'efinition}
